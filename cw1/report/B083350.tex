% !TEX options=--shell-escape

\documentclass[11pt]{article}

\usepackage{geometry}
\usepackage[hidelinks, colorlinks=true, linkcolor=blue, citecolor=blue]{hyperref}

% APA citation style
\usepackage{natbib}
\usepackage{apalike}

\usepackage{setspace} % line spacing

% Paragraph indentation (first line)
\setlength{\parindent}{0em}
% Paragraph spacing
\setlength{\parskip}{.7em}

% Custom horizontal rules in tables
% \usepackage{booktabs}

% multi-row table cells
% \usepackage{multirow}

% \usepackage{subfigure}
% \usepackage{subcaption}
\usepackage{caption}

% Names of subsections and figures in cross-references with \autoref{}
\renewcommand*{\sectionautorefname}{Section}%
\renewcommand*{\subsectionautorefname}{Section}%
\renewcommand*{\figureautorefname}{Fig.}%

\usepackage{textcomp} % fancy symbols in text mode (e.g. right arrow)
\newcommand{\textapprox}{\raisebox{0.5ex}{\texttildelow}}

% For better enums, e.g. resuming after a normal line of text
% \usepackage{enumitem}

% For images
\usepackage{graphicx, rotating}

% Center caption text for images
\usepackage[justification=centering]{caption}

% styling of quotes (use like: \Quote{text}{label}), including their counting and referencing with \qref{label}
\newcounter{quotecount}
\newcommand{\Quote}[2]{\vspace{0.4cm}\\ 
	\hspace*{1cm}\parbox{12cm}{\em #1 \refstepcounter{quotecount}\label{#2}}\hspace*{1cm}(\arabic{quotecount})\\[0.4cm]}
\newcommand{\qref}[1]{(\autoref{#1})}


% \onehalfspacing
\title{\LARGE Introduction to Discourse Analysis -- Assignment 1}
\date{}

\begin{document}

\maketitle

\,
\vspace{-5em}

%%%%%%%%%%%%%%%%%%%%%
% total words: 1553 %
%%%%%%%%%%%%%%%%%%%%%

% 173
\section{Introduction}{
	% - Nur & Ruijuan analyse the ways in which 2 successful political speeches achieve this
	% - in particular, they analyse the interpersonal metafunction (proposed by Halliday in context of SFL) of the language, focusing on mood modality and pronoun
	% - they observe that the most dominant tools employed are positive declaratives, modals with high degree of modal committment, plural 1st person pronouns
	% - they also notice less frequent but effective careful use of imperatives, 2nd & 3rd person pronouns
	% - to see how their findings generalise, I apply their approach to Obama's inauguration speech

	% - political speech is a tool and does this, this, that
	Political speeches go beyond content, they are tools for influencing listeners, 
	% - a successful speech will bridge gaps, make audience identify with the message, believe and support the speaker
	bridging the speaker-audience gap, making listeners identify with and support the speaker. 
	% - IM is great for analysing these things
	To understand speeches from this perspective, one can analyse the Interpersonal Metafunction of language, proposed by \citet{Halliday1970IM}.
	% - Nur & Ruijuan analyse IM in political speeches, on Mandela's 1994 inauguration and Obama's 2009 victory speech
	\citet{Ye} and \citet{Nur} apply such analysis to Barack Obama's 2008 election victory speech and Nelson Mandela's 1994 inauguration speech, respectively.
	% - they analyse the same elements of IM, and they come to very similar conclusions w.r.t use of and role of IM elements in the two speeches
	Observing that their conclusions are very similar, 
	% - I challenge this, hypothesising that actually differences between inauguration and victory speeches could surface in the use of IM
	I challenge this agreement and hypothesise that victory and inauguration speeches differ in ways that should surface in Interpersonal Metafunction.
	% - victory is for supporters who already support the person, while inauguration is for everyone and needs to unite while presenting plans
	While victory speeches typically address supporters, celebrate and reflect on campaigns, inauguration speeches present priorities of upcoming presidencies and aim at all citizens, trying to unite them behind the proposed goals.
	% - I chose Obama's inauguration so that as many properties as possible stay the same (speaker, time period, affiliation, etc) and only the occasion changes
	To verify my hypothesis, I analyse the 2009 inauguration speech of Barack Obama, which shares the time period and speaker with that analysed by  \citeauthor{Ye}. 
	% - I'll explore the same IM elements Ruijuan did and will compare with Ruijuan's findings
	Thus, relative to \citeauthor{Ye}, I keep most variables fixed while varying the occasion, audience and ultimately aims of speech. 
	After the analysis, I juxtapose my findings with those of \citeauthor{Ye} and \citeauthor{Nur}.
}

% 148
\section{Theoretical background}{
	% - Halliday developed SFL
	% - language is functionally organised, it's shaped by the functions that is serves
	In Systemic Functional Linguistics, Halliday views language as shaped by functions it serves -- functionally organised. The different functions then relate to different kinds of meaning language carries.
	% - Halliday describes 3 such metafunctions and argues that, subsequently, the meaning of a language can be explained from the corresponding 3 different perspectives (_strands of meaning_):
	The (meta)functions \citet{Halliday1970IM} proposes, with corresponding meaning types, are:
	\begin{enumerate}
	%   - **experiential** metafunction as use of language for talking about reality, human experience of the world, etc., constituting **ideational** meaning
		\item \textbf{experiential metafunction} -- as an ``observer's'' tool for describing reality, experience, ideas; leading to \textbf{ideational meaning},
	%   - **interpersonal** metafunction -- language as tool/medium for social interaction; giving rise to the **interpersonal** (social) meaning concerned with interactions, relationships, etc
		\item \textbf{interpersonal metafunction} -- as an ``actor's'' tool for taking a stance or role, interacting with and influencing others; creating \textbf{interpersonal meaning},
	%   - **textual** metafunction -- language as a communicative tool, a message -- the surface form, constructed so as to fit well with the local context (preceding messages) as well as wider (e.g. social) context of the message, leading to the **textual** meaning about language as a message
		\item \textbf{textual metafunction} -- as a ``communicator's'' tool for making the previous two meanings surface coherently, considerate of wider social and situational context and local context of surrounding text; leading to \textbf{textual meaning}.
	\end{enumerate}
	% - the 3 metafunctions are intertwined, but the interpersonal one is the most important one in political speeches as it describes how language can be employed as a tool to bring the speaker closer to their audience (lessen the _social distance_), establish a positive or close relationship, influence, persuade, inspire the audience, etc.	
	The different functions co-exist in language, but the interpersonal metafunction is what makes language a social instrument like in political speeches. Speakers use elements of the metafunction to assume role, change the social distance between them and audiences, establish relationships, unite or divide, inspire and persuade.
}

% 103
\section{Data}{
	% - january 2009, 2.5 months after victory
	Obama's inauguration speech took place in January 2009 (2.5 months after the victory speech).
	% - context of global financial crisis and war in Iraq and environmental challenges
	America was facing the global financial crisis and war in Iraq, and an approaching environmental crisis was becoming clearer.
	% - focus on facing the difficulties (like ancestors did)
	Obama names these directly and, by recalling the difficulties past generations of Americans faced with perseverance and high spirit, 
	% - it aims to unite all Americans and present US as a world leader
	motivates citizens to unite and fight again.
	% differences: occasion, audience, goals
	Addressed to all Americans and other nations, the speech tried to win support for the upcoming presidency's challenging goals.
	% similarities
	Because the speaker stays the same, with the same affiliation as in the victory speech, I hypothesise that any observed differences relate mainly to the different situations.

	% - I chose Obama's 2009 inauguration speech because I expect interesting similarities and differences with previous speeches:
	%   - at the basic level, it's a presidential speech (like the others)
	%   - as an inauguration speech, it could be close (=> better comparable) to Mandela's speech more than the victory speech analysed by Ruijuan
	%   - as a speech by Obama, and coming only 2.5 months after his victory speech, I expect it to be similar
	%   - on the other hand, differences with Obama2008 can be related to the different occasions (election victory vs inauguration), audiences and purposes:
	%     - while the victory speech was aimed primarily for Obama's supporters, the inaugural speech was widely televised and addressed to all Americans and even to other nations
}

% 991
\section{Analysis}{
	% 19
	%  - I analyse the dimensions of IM explored by both Nur and Ruijuan, i.e. mood, modality and personal pronouns.
	For comparability with \citeauthor{Nur} and \citeauthor{Ye}, I analyse the same elements of Interpersonal Metafunction: mood, modality and personal pronouns.
 	
	% 300 - 9(table)
 	%  - Mood
 	\subsection{Mood}{
		%    - The concept of Mood relates to the very nature of clause as exchange
 		Language interaction can be viewed as speaker-initiated exchange of commodity: The speaker gives or demands, the commodity being information or goods/services.
		%    - provides basis for interactions between addresser and audience in the form of 4 basic speech functions as exchanges of 2 types of commodity: 
 		The mood system is then ``the expression of the speaker's choice of role in the communicative situation'' \citep{Halliday1970ModalityMood}, where choosing to give/demand information leads to statements/questions, and giving/demanding goods-and-services leads to offers/commands. While choosing commands easily widens the speaker-audience social gap, statements or offers can achieve the opposite.
		%    - the mood itself is in a clause represented by subject + finite (typically the subject-verb pair)
		% statement (giving information), question (demanding information), offer (giving goods & services), and command (demanding g&s)

		%    - declarative/imperative/interrogative
		%      - observed: 106/9/0: 92/8/0%
		%      - Ruijuan: 101/5/2: 94/5/1%
		%      - Nur: 36/5/0: 88/12/0%
		\begin{table}[h!tb]
	      \centering
			\begin{tabular}{l|ccc}
			         & declarative & imperative & interrogative \\
			  \hline
			  \hline
			  here   & 92\%        & 8\%        & 0\% \\
			  \hline
			  Ye     & 94\%        & 5\%        & 1\% \\
			  Nur    & 88\%        & 12\%       & 0\% \\
			\end{tabular}
	      \caption{Clause types.}
	      \label{tab:mood}
	    \end{table}

 		Grammatically, statements are realised as declarative clauses, questions as interrogatives, and commands as imperatives. \autoref{tab:mood} shows that Obama uses almost exclusively declaratives -- assumes the role of information giver. Thus, he effectively communicates values that unite all Americans:
		%    - declarative examples
		\Quote{On this day, we gather because we have chosen hope over fear, unity of purpose over conflict and discord.}{declarative1}
 		but also persuades and motivates:
 		\Quote{Today I say to you that the challenges we face are real.}{declarative2}
 		\Quote{Starting today, we must pick ourselves up, dust ourselves off, and begin again the work of remaking America.}{declarative3}
 		while still sounding strong and relatable, not confrontational or patronising.

		%    - imperatives addressed to "others" countries and potential enemies (3x), and to all Americans (the Let form, 3x)
		%    - imperative example:
 		The few imperatives serve two roles. They demonstrate strength and courage when confronting enemies:
 		\Quote{\normalfont To those leaders around the globe who seek to sow conflict, or blame their society's ills on the West,\em\  know that your people will judge you on what you can build, not what you destroy.}{imperative1}
 		In the remaining cases -- when addressing Americans -- imperatives are used not to command, but instead use ``let'' to emotionally encourage listeners to join the action:
 		\Quote{With hope and virtue, let us brave once more the icy currents, and endure what storms may come.}{imperative2}
		%    - similar to previous findings. more imperatives in inaugural vs victory speeches may be attributed to presenting program and encouraging people to join and act in accordance with it
		Altogether, declaratives and imperatives positively assert key messages, showing Obama as a strong, protective leader who invites all citizens to join him.
 	}

 	% 289 - 27(table)
 	%  - Modality
 	\subsection{Modality}{
 		Modality refers to intermediate polarities from the ``yes''--``no'' continuum.
		%    - covering the range of meanings between yes and no, modality refers to how speaker's position, (un)certainty, (in)ability and attitude surface in semantic meaning
		%    - one can go further and distinguish between:
 		The system enables:
 		\begin{itemize}
		%      - modalization as speaker's judgement of proposition in terms of probablity (possible -> certain) and usuality (sometimes->always)
 			\item speakers to assume positions towards a proposition, in terms of its probability and usuality (termed \textit{modalization})
		%      - modulation as expressing speaker's sense of obligation (allowed->required) and inclination (willing->determined)
 			\item describing levels of obligation and inclination (termed \textit{modulation})
 		\end{itemize}
		%      - but Halliday 1970 argues that these are "the same system in different functions", whereas modalization changes the meaning at the interpersonal level while modulation changes the content of the proposition at the ideational level
		%    - both can be realised in various forms, using e.g. modal verbs, adjectives, adverbs...
		%    - I limit myself to modal verbs, analysed in Nur and Ruijuan
 		While modulation happens at ideational (content) level and modalization at interpersonal level (speaker expressing their stance), they both surface primarily as modal verbal operators (``can'', ``must'', ...). \citet{Halliday1970ModalityMood} notes that the ideational part of modality is \textit{oriented towards} the interpersonal component and vice versa; modulation and modalization are ``the same system in different functions''.
		%                | will/won't (future/wish) | would | must | can/cannot | could | be able to | other | total
		%    - observed: | 19         (13/6)        | 1     | 8(6) | 12/6       | 0     | 0          | 8     | 54 might(2) shall(3) may(3)
		%    - Ruijuan:  | 18         (7/11)        | ?     | 4    | 22         | ?     | ?          | 11    | 55
		%    - Nur:      | 6          (?/?)         | ?     | 3    | 1          | 1     | 1          | 3     | 15
		\begin{table}[h!tb]
	      \centering
	        \begin{tabular}{l|cccccccc}
	                & will (future/wish) & would & must & can/cannot & could & be able to & other & total \\
	          \hline
	          \hline
	          here  & 35\%   (24\%/11\%) & 2\%   & 15\% & 22\%/11\%  & 0\%   & 0\%        & 15\%  & 54 \\
	          \hline
	          Ye    & 33\%   (13\%/20\%) & ?     & 7\%  & 40\%       & ?     & ?          & 20\%  & 55 \\
	          Nur   & 40\%   (?/?)       & ?     & 20\% & 7\%        & 7\%   & 7\%        & 20\%  & 15 \\
	        \end{tabular}
	      \caption{Modal verb operators.}
	      \label{tab:modality}
	    \end{table}

		%    - Most dominant is the high commitment (will, must), but note that of 19 wills only 6 communicate strong will/determination, while 13 are simply markers of future tense. also, I identify 2 cases of must which I don't see as communicating urge, but simply obligation:
	    \autoref{tab:modality} shows that the strongest modals (will, must) dominate the inauguration speech, but, looking carefully, only 1/3 of ``will''s express determination/strong wish:
	    \Quote{All this we can do.\\And all this we will do.}{will1}
	    while the remainder are ``will'' used merely as future tense marker in presenting Obama's plans:
	    \Quote{Where the answer is no, programs will end.}{will2}
	    Similarly with ``must'': In 6 cases it communicates urge and obligation like in \qref{declarative3}, but in 2 cases carries no strong charge:
	    \Quote{\normalfont [...] greatness is never a given.\\ \em It must be earned.}{must1}
	    Overall, the high commitment modals successfully convey Obama's determination in an emotional way, but also many of them don't contribute to the interpersonal metafunction.

		%    - can is also dominant, BUT used only in 2x cases to strongly communicate the ability (in all other cases it's either negative, or communicating in a rather ordinary way the possibility/ability) -- note that Obama here doesn't use 'yes we can':
	    The frequent ``can'' must also be analysed carefully. I find only 2 instances where ``can'' emotionally persuades audience about their abilities; see \qref{will1} and:
	    \Quote{Guided by these principles once more, we can meet those new threats that demand even greater effort [...]}{can1}
	    the remaining 16 instances used without much interpersonal effect to express ability/possibility, see \qref{imperative1} and:
	    \Quote{For as much as government can do and must do [...]}{can2}
	}

	% 196 240 - 27(table)
	\subsection{Personal pronouns}{
		By choosing pronouns, political speaker can establish different speaker-audience relations; e.g. ``I'' emphasises the \textit{I-you} gap,``we'' can bridge it, and ``they'' emphasises the \textit{we-they} gap.

		%  - Personal pronoun (everything in percent)
		%    -          | I  | We (excl./incl.) | Us | Our | You | Your | S/he | They/Their/them/it | total
		%    - observed | 1  | 27 (4/23)        | 10 | 30  | 6   | 1    | 0.5  | 8/4/2/8            | 224
		%    - Ruijuan  | 19 | 31               | 8  | 17  | 10  | 3    | 10   | 2/?/?/?            | 131
		%    - Nur      | 1  | 34               | 11 | 26  | 3   | 0    | 0    | 1/9/3/13           | 80
		\begin{table}[h!tb]
	      \centering
	        \begin{tabular}{l|ccccccccc}
	                & I    & we (excl./incl.) & us   & our  & you  & your & s/he  & they/their/them/it & total \\
	          \hline
	          \hline
	          here  & 1\%  & 27\% (4\%/23\%)  & 10\% & 30\% & 6\%  & 1\%  & 0.5\% & 8\%/4\%/2\%/8\%    & 224 \\
	          \hline
	          Ye    & 19\% & 31\%             & 8\%  & 17\% & 10\% & 3\%  & 10\%  & 2\%/?/?/?          & 131 \\
	          Nur   & 1\%  & 34\%             & 11\% & 26\% & 3\%  & 0\%  & 0\%   & 1\%/9\%/3\%/13\%   & 80 \\
	        \end{tabular}
	      \caption{Personal pronouns.}
	      \label{tab:pronouns}
	    \end{table}

		%    - 1st person dominates, exclusively plural: aiming to unite people and portray Obama as one of them (compare with victory speech where I is used often). we is found in 53% of all main clauses (compare with 38% in victory and 66% in Mandela -- this shows how inauguration is different from victory)
	   	\autoref{tab:pronouns} shows that Obama uses overwhelmingly the 1st-person plural forms, thus portraying all Americans as united (and him as one of them), see \qref{declarative1}, \qref{declarative2}, \qref{declarative3}, \qref{imperative2}, \qref{will1}, \qref{can1}. Only 1/6 of all ``we''s employs the exclusive, dividing \textit{we all vs they} sense -- mainly relating to enemies:
	   	\Quote{[...] for those who seek to advance their aims by inducing terror and slaughtering innocents, we say to you now that our spirit is stronger and cannot be broken; you cannot outlast us, and we will defeat you.}{we1}
		%    - 2nd person used almost exclusively to address (confront) 3rd parties such as foreign countries (including enemies), setting a strong (and somewhat patronising) tone
	   	This directly confrontational tone is also the main way in which Obama uses 2nd-person pronouns (in 14/17 cases) -- rather then opting for the more indirect confrontation using 3rd person. Only in 3 cases does he use ``you'' to address Americans: when using ``I-you'' to establish an intimate bond and express humbleness:
	   	\Quote{My fellow citizens: I stand here today humbled by the task before us, grateful for the trust you have bestowed [...]}{you1}
		%    - 3rd person (mainly they/their) is used to refer to those not present, mainly to earlier generations who worked/fought hard so the present-day Americans could live decent lives, and to American soldiers. By referring to these shared American values and symbols, Obama unites the audience.
		3rd-person pronouns Obama uses to talk of those not present whom he doesn't want to confront, but respect: past generations and US troops. Thus, the speech is grounded in values widely shared by Americans:
		\Quote{For us, they fought and died, in places like Concord and Gettysburg; Normandy and Khe Sahn.}{they1}
	}
}

% 295
\section{Comparing with Nur and Ye}{
	\label{sec:comparison}
	% afaik, both explore how IM helps achieve aims of the speech
	\citeauthor{Nur} and \citeauthor{Ye} analyse how Interpersonal Metafunction helps speeches succeed.
	% but the speeches have different aims (I think), and hence once can expect different uses of IM
	% I reason that victory and inauguration speeches have different aims and hence one should expect different uses of the metafunction for success.
	% this isn't visible in the conclusions! but it should be because IM is what makes speeches good tols, right?
	Looking at their main conclusions, I see no principal differences.
	% so, if their findings are to be generalizable (in that we can continue to use IM to explain how successful speeches work), I wanna be able to also see differences
	However, to confirm generalizability of their results (and hence usefulness of Interpersonal Metafunction for analysing political speeches), I need to see that findings reflect the different aims of inauguration and victory speeches.

	%    - **MOOD: previous findings**:
	%      - Nur: 
	%        - declarative clauses dominate {_making the speech solemn and persuasive, thrivingly recalling Mandela and his countrymen’s long sufferings, expressing gratitude to cowarriors and supporters and promising to work for the actual freedom_}
	%        - followed by imperatives (non-commanding Let) {_helping Mandela arouse people’s passion to dream and act together to earn equality & freedom; minimizes the social distance; making speech more emotive, appealing, and inspiring._}
	%      - Ruijuan:
	%        - declaratives dominate {_give as much as possible information to the audience, recalling election campaign, expressing gratitude to supporters, promising, inspiring the audience to face difficulties as one_}
	%        - imperatives (Let) follow {_shorten distance, call to act together & overcome the difficulties.; make speech more moving, appealing and inspiring_}
	
	% same as my findings, numbers agree, everything OK
	Regarding mood, our findings hugely agree: Declaratives dominate, communicating ideas and values, uniting speakers with listeners, and inspiring. The (non-commanding) infrequent imperatives follow, adding emotions, encouraging people to face difficulties/enemies.

	%    - **MODALITY: previous findings**
	%      - Nur
	%        - will dominates (as future predicator & symbol of determination)
	%        - must follows {_conveys strong determination; calls audience to be determined to act towards common objectives_}
	%        - scarse can {_encourages people to believe in their abilities to act together; minimizes gap_}
	%      - Ruijuan
	%        - can dominates {_weaken authority & shorten distance. 'Be able' showed in repeated “Yes we can”, encourages people to believe in themselves & ability to climb back into the light_}
	%        - will follows {_shows strong mind and keen desire to lead through the difficulties; higher modal commitment confirms more actions will be definitely taken in the future_}
	%        - must follows {_shows firm determination to overcome difficulties, call to actions to achieve targets_}
	
	% different frequencies in must (but conclusions same!); I say that inauguration => more musting, also checks with Nur
	In modality, numbers show more frequent ``must'' in inauguration speeches (\autoref{tab:modality}), yet the conclusions seem identical: that ``must'' communicates determination and calls to action.
	% I see that inauguration => more future will, less wishing will (cannot confirm with Nur...)
	Likewise, Ye's and my conclusions about ``will'' are close, yet the numbers show ``will'' communicating mainly strong determination in Obama's victory speech while mostly describing future actions in Obama's inauguration.
	The remaining conclusions and statistics mostly agree well (although the frequency of ``can'' in Ye is skewed by repetitions of \textit{yes we can}).
	% shows how repetition can skew numbers: yes we can repeated 6x
	Altogether, I see that inauguration speeches use ``must'' and the future ``will'' more frequently, which I relate to proposing/motivating plans for new presidency. 


	%    - **PRONOUNS: previous findings**
	%      - Nur
	%        - 1st: plural replaces singular {_inclusive kindles to feel same as Mandela, come forward & work together. exclusive implies authority & power. government integrated with people, grateful, with high spirit and powerful, ready to protect and unite._}
	%        - 2nd: least frequent {_crosses audience’s boundary, shows humble gratitude to international body_}
	%        - 3rd: frequent {_shows care/respect to people who died during the struggle, invites audience to feel same. Ties speaker+audience in intimate bondage_}
	%      - Ruijuan
	%        - 1st: most frequent, mainly plural, {_'I' makes Obama sincere person who remembers gratitude and will repay it. Inclusive 'we' makes Obama on the same boat as audience & persuades to share his proposals. Exclusive 'we' makes them strong leaders; wins Americans’ confidence in new government._}
	%        - 2nd: infrequent {_“you” attracts attention + creates dialogical feel; Obama shows care/respect to audience => intimate relation_}
	%        - 3rd: infrequent

	% MISMATCH in use of I, not visible from conclusions!!! shows importance of comparing numbers
	In personal pronouns, our findings and statistics agree, with a few notable exceptions. 
	% - Ye: Obama uses I ``to speak of his election campaign and express his gratitude. '', `` application of “I” here successfully describes the new elected president into a sincere person who will remember the gratitude and try to repay it''
	``I'' is much less frequent in inauguration speeches, although I very similarly to \citeauthor{Ye} conclude that ``I'' portrays Obama as a close person ready to repay his gratitude.
	% - Ye also has more ``you'' which is connected to using ``I and you'', not we
	The more frequent ``you'' in the victory speech I explain as a correlate of ``I'' (the ``I-you'' construction, as opposed to ``we'') and relate it to the elect personally reflecting on campaign, while the inauguration is more about the collective journey ahead.
	% - s/he shows how small things can skew stats: in this case repeating things about Ann Nixon Cooper (most of these pronouns refer to her)
	Finally, the frequent use of ``she'' in the victory speech is another example of a skewed number; here because of repeated referring to an anecdote involving Ann Nixon Cooper.
}

% 93
\section{Conclusions}{
	% mostly agreeing conclusions as Nur and Ye
	While my results mostly agree with those of \citeauthor{Ye} and \citeauthor{Nur},
	% point out differences between Nur+Ye, and also with my analysis, and explain and relate to differences in inauguration vs victory aims
	I am also able to relate the few differences to the different aims of victory vs inauguration speech. 
	% thus, I confirm generalizability of Nur and Ye's results in that my results agree and IM can explain and account for differences in aims
	Thus, I confirm generalizability of \citeauthor{Ye} and \citeauthor{Nur}'s results in that Interpersonal Metafunction usefully sheds light into structures behind successful speeches, while similarities and differences in results reflect the differences stemming from the speeches' situational and/or social context.
	% I also warn that conclusions should reflect stats unless there is suspicion that stats are skewed
	Importantly, I advocate for combining qualitative and quantitative findings in forming conclusions, illustrating how numbers alone can be skewed while qualitative results can hide important statistical differences.
}

\bibliographystyle{apalike}
\bibliography{B083350.bib}

\section*{Appendix -- Obama's 2009 inauguration speech}{
(source: \url{https://avalon.law.yale.edu/21st_century/obama.asp})

My fellow citizens:

I stand here today humbled by the task before us, grateful for the trust you have bestowed, mindful of the sacrifices borne by our ancestors. I thank President Bush for his service to our nation, as well as the generosity and cooperation he has shown throughout this transition.

Forty-four Americans have now taken the presidential oath. The words have been spoken during rising tides of prosperity and the still waters of peace. Yet, every so often the oath is taken amidst gathering clouds and raging storms. At these moments, America has carried on not simply because of the skill or vision of those in high office, but because We the People have remained faithful to the ideals of our forbearers, and true to our founding documents.

So it has been. So it must be with this generation of Americans. That we are in the midst of crisis is now well understood. Our nation is at war, against a far-reaching network of violence and hatred. Our economy is badly weakened, a consequence of greed and irresponsibility on the part of some, but also our collective failure to make hard choices and prepare the nation for a new age. Homes have been lost; jobs shed; businesses shuttered. Our healthcare is too costly; our schools fail too many; and each day brings further evidence that the ways we use energy strengthen our adversaries and threaten our planet.

These are the indicators of crisis, subject to data and statistics. Less measurable but no less profound is a sapping of confidence across our land; a nagging fear that America's decline is inevitable, and that the next generation must lower its sights.

Today I say to you that the challenges we face are real. They are serious and they are many. They will not be met easily or in a short span of time. But know this, America: they will be met.

On this day, we gather because we have chosen hope over fear, unity of purpose over conflict and discord.

On this day, we come to proclaim an end to the petty grievances and false promises, the recriminations and worn out dogmas, that for far too long have strangled our politics.

We remain a young nation, but in the words of Scripture, the time has come to set aside childish things. The time has come to reaffirm our enduring spirit; to choose our better history; to carry forward that precious gift, that noble idea, passed on from generation to generation: the God-given promise that all are equal, all are free, and all deserve a chance to pursue their full measure of happiness.

In reaffirming the greatness of our nation, we understand that greatness is never a given. It must be earned. Our journey has never been one of shortcuts or settling for less. It has not been the path for the faint-hearted, for those who prefer leisure over work, or seek only the pleasures of riches and fame. Rather, it has been the risk-takers, the doers, the makers of things - some celebrated, but more often men and women obscure in their labor, who have carried us up the long, rugged path towards prosperity and freedom.

For us, they packed up their few worldly possessions and traveled across oceans in search of a new life.

For us, they toiled in sweatshops and settled the West; endured the lash of the whip and plowed the hard earth.

For us, they fought and died, in places like Concord and Gettysburg; Normandy and Khe Sahn.

Time and again these men and women struggled and sacrificed and worked till their hands were raw so that we might live a better life. They saw America as bigger than the sum of our individual ambitions; greater than all the differences of birth or wealth or faction.

This is the journey we continue today. We remain the most prosperous, powerful nation on Earth. Our workers are no less productive than when this crisis began. Our minds are no less inventive, our goods and services no less needed than they were last week or last month or last year. Our capacity remains undiminished. But our time of standing pat, of protecting narrow interests and putting off unpleasant decisions -- that time has surely passed. Starting today, we must pick ourselves up, dust ourselves off, and begin again the work of remaking America.

For everywhere we look, there is work to be done. The state of the economy calls for action, bold and swift, and we will act not only to create new jobs, but to lay a new foundation for growth. We will build the roads and bridges, the electric grids and digital lines that feed our commerce and bind us together. We will restore science to its rightful place, and wield technology's wonders to raise healthcare's quality and lower its cost. We will harness the sun and the winds and the soil to fuel our cars and run our factories. And we will transform our schools and colleges and universities to meet the demands of a new age. All this we can do. And all this we will do.

Now, there are some who question the scale of our ambitions -- who suggest that our system cannot tolerate too many big plans. Their memories are short. For they have forgotten what this country has already done; what free men and women can achieve when imagination is joined to common purpose, and necessity to courage.

What the cynics fail to understand is that the ground has shifted beneath them - that the stale political arguments that have consumed us for so long no longer apply. The question we ask today is not whether our government is too big or too small, but whether it works - whether it helps families find jobs at a decent wage, care they can afford, a retirement that is dignified. Where the answer is yes, we intend to move forward. Where the answer is no, programs will end. And those of us who manage the public's dollars will be held to account - to spend wisely, reform bad habits, and do our business in the light of day - because only then can we restore the vital trust between a people and their government.

Nor is the question before us whether the market is a force for good or ill. Its power to generate wealth and expand freedom is unmatched, but this crisis has reminded us that without a watchful eye, the market can spin out of control and that a nation cannot prosper long when it favors only the prosperous. The success of our economy has always depended not just on the size of our Gross Domestic Product, but on the reach of our prosperity; on our ability to extend opportunity to every willing heart - not out of charity, but because it is the surest route to our common good.

As for our common defense, we reject as false the choice between our safety and our ideals. Our Founding Fathers, faced with perils we can scarcely imagine, drafted a charter to assure the rule of law and the rights of man, a charter expanded by the blood of generations. Those ideals still light the world, and we will not give them up for expedience's sake. And so to all other peoples and governments who are watching today, from the grandest capitals to the small village where my father was born: know that America is a friend of each nation and every man, woman, and child who seeks a future of peace and dignity, and that we are ready to lead once more.

Recall that earlier generations faced down fascism and communism not just with missiles and tanks, but with sturdy alliances and enduring convictions. They understood that our power alone cannot protect us, nor does it entitle us to do as we please. Instead, they knew that our power grows through its prudent use; our security emanates from the justness of our cause, the force of our example, the tempering qualities of humility and restraint.

We are the keepers of this legacy. Guided by these principles once more, we can meet those new threats that demand even greater effort, even greater cooperation and understanding between nations. We will begin to responsibly leave Iraq to its people, and forge a hard-earned peace in Afghanistan. With old friends and former foes, we will work tirelessly to lessen the nuclear threat, and roll back the specter of a warming planet. We will not apologize for our way of life, nor will we waver in its defense, and for those who seek to advance their aims by inducing terror and slaughtering innocents, we say to you now that our spirit is stronger and cannot be broken; you cannot outlast us, and we will defeat you.

For we know that our patchwork heritage is a strength, not a weakness. We are a nation of Christians and Muslims, Jews and Hindus, and non-believers. We are shaped by every language and culture, drawn from every end of this Earth; and because we have tasted the bitter swill of civil war and segregation, and emerged from that dark chapter stronger and more united, we cannot help but believe that the old hatreds shall someday pass; that the lines of tribe shall soon dissolve; that as the world grows smaller, our common humanity shall reveal itself; and that America must play its role in ushering in a new era of peace.

To the Muslim world, we seek a new way forward, based on mutual interest and mutual respect. To those leaders around the globe who seek to sow conflict, or blame their society's ills on the West, know that your people will judge you on what you can build, not what you destroy. To those who cling to power through corruption and deceit and the silencing of dissent, know that you are on the wrong side of history; but that we will extend a hand if you are willing to unclench your fist.

To the people of poor nations, we pledge to work alongside you to make your farms flourish and let clean waters flow; to nourish starved bodies and feed hungry minds. And to those nations like ours that enjoy relative plenty, we say we can no longer afford indifference to suffering outside our borders; nor can we consume the world's resources without regard to effect. For the world has changed, and we must change with it.

As we consider the road that unfolds before us, we remember with humble gratitude those brave Americans who, at this very hour, patrol far-off deserts and distant mountains. They have something to tell us today, just as the fallen heroes who lie in Arlington whisper through the ages. We honor them not only because they are guardians of our liberty, but because they embody the spirit of service: a willingness to find meaning in something greater than themselves. And yet, at this moment, a moment that will define a generation, it is precisely this spirit that must inhabit us all.

For as much as government can do and must do, it is ultimately the faith and determination of the American people upon which this nation relies. It is the kindness to take in a stranger when the levees break, the selflessness of workers who would rather cut their hours than see a friend lose their job, which sees us through our darkest hours. It is the firefighter's courage to storm a stairway filled with smoke, but also a parent's willingness to nurture a child, that finally decides our fate.

Our challenges may be new. The instruments with which we meet them may be new. But those values upon which our success depends - hard work and honesty, courage and fair play, tolerance and curiosity, loyalty and patriotism - these things are old. These things are true. They have been the quiet force of progress throughout our history. What is demanded then is a return to these truths. What is required of us now is a new era of responsibility - a recognition, on the part of every American, that we have duties to ourselves, our nation, and the world, duties that we do not grudgingly accept but rather seize gladly, firm in the knowledge that there is nothing so satisfying to the spirit, so defining of our character, than giving our all to a difficult task.

This is the price and the promise of citizenship.

This is the source of our confidence: the knowledge that God calls on us to shape an uncertain destiny.

This is the meaning of our liberty and our creed, why men and women and children of every race and every faith can join in celebration across this magnificent mall, and why a man whose father less than 60 years ago might not have been served at a local restaurant can now stand before you to take a most sacred oath.

So let us mark this day with remembrance, of who we are and how far we have traveled. In the year of America's birth, in the coldest of months, a small band of patriots huddled by dying campfires on the shores of an icy river. The capital was abandoned. The enemy was advancing. The snow was stained with blood. At a moment when the outcome of our revolution was most in doubt, the father of our nation ordered these words be read to the people:

"Let it be told to the future world that in the depth of winter, when nothing but hope and virtue could survive that the city and the country, alarmed at one common danger, came forth to meet it."

America. In the face of our common dangers, in this winter of our hardship, let us remember these timeless words. With hope and virtue, let us brave once more the icy currents, and endure what storms may come. Let it be said by our children's children that when we were tested we refused to let this journey end, that we did not turn back, nor did we falter; and with eyes fixed on the horizon and God's grace upon us, we carried forth that great gift of freedom and delivered it safely to future generations.

Thank you. God bless you.

And God bless the United States of America.
}

\end{document}
