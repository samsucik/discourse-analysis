% !TEX options=--shell-escape

\documentclass[11pt]{article}

\usepackage{geometry}
\usepackage[hidelinks, colorlinks=true, linkcolor=blue, citecolor=blue]{hyperref}

% APA citation style
\usepackage{natbib}
\usepackage{apalike}

\usepackage{setspace} % line spacing

% Paragraph indentation (first line)
\setlength{\parindent}{0em}
% Paragraph spacing
\setlength{\parskip}{.7em}

% Custom horizontal rules in tables
% \usepackage{booktabs}

% multi-row table cells
% \usepackage{multirow}

% \usepackage{subfigure}
% \usepackage{subcaption}
\usepackage{caption}

% Names of subsections and figures in cross-references with \autoref{}
\renewcommand*{\subsectionautorefname}{Section}%
\renewcommand*{\figureautorefname}{Fig.}%

\usepackage{textcomp} % fancy symbols in text mode (e.g. right arrow)
\newcommand{\textapprox}{\raisebox{0.5ex}{\texttildelow}}

% For better enums, e.g. resuming after a normal line of text
% \usepackage{enumitem}

% For images
\usepackage{graphicx, rotating}

% Center caption text for images
\usepackage[justification=centering]{caption}

% \onehalfspacing
\title{\LARGE Introduction to Discourse Analysis -- Assignment 1}
\date{}

\begin{document}

\maketitle

\,
\vspace{-5em}

\section*{Introduction}{
	% - Nur & Ruijuan analyse the ways in which 2 successful political speeches achieve this
	% - in particular, they analyse the interpersonal metafunction (proposed by Halliday in context of SFL) of the language, focusing on mood modality and pronoun
	% - they observe that the most dominant tools employed are positive declaratives, modals with high degree of modal committment, plural 1st person pronouns
	% - they also notice less frequent but effective careful use of imperatives, 2nd & 3rd person pronouns
	% - to see how their findings generalise, I apply their approach to Obama's inauguration speech

	% - political speech is a tool and does this, this, that
	Political speeches are more than their content, they are tools used to influence listeners and/or diminish the opposition.
	% - a successful speech will bridge gaps, make audience identify with the message, believe and support the speaker
	A successful speech will bridge the gap between the speaker and the audience, making listeners identify with the message, believe and support the politician.
	% - IM is great for analysing these things
	To analyse speeches from this perspective, one can turn to Interpersonal Metafunction, one of the functions of language proposed by \citet{Halliday1970IM} as part of his functional view of language.
	% - Nur & Ruijuan analyse IM in political speeches, on Mandela's 1994 inauguration and Obama's 2009 victory speech
	\citet{Ye} and \citet{Nur} analyse Interpersonal Metafunction in Barack Obama's 2008 presidential election victory speech and in Nelson Mandela's 1994 inauguration speech, respectively, 
	% - they analyse the same elements of IM, and they come to very similar conclusions w.r.t use of and role of IM elements in the two speeches
	arriving at very similar conclusions with respect to the use and role of elements of the metafunction in the two speeches.
	% - I challenge this, hypothesising that actually differences between inauguration and victory speeches could surface in the use of IM
	I challenge this finding and hypothesise that victory and inauguration speeches differ in ways that should surface in Interpersonal Metafunction.
	% - victory is for supporters who already support the person, while inauguration is for everyone and needs to unite while presenting plans
	While victory speeches address supporters of the elect and will likely reflect on the successful campaign and celebrate, inauguration speeches present the program of the upcoming presidency and are aimed at the entire nation, thus needing to unite all citizens and make them believe and support the set common goals.
	% - I chose Obama's inauguration so that as many properties as possible stay the same (speaker, time period, affiliation, etc) and only the occasion changes
	To verify my hypothesis, I analyse a speech that shares the time period and speaker with that analysed by \citeauthor{Ye}: the 2009 inauguration speech of Barack Obama. 
	% - I'll explore the same IM elements Ruijuan did and will compare with Ruijuan's findings
	This way, and by analysing the same elements of Interpersonal Metafunction as \citeauthor{Ye}, I keep many variables fixed while varying the occasion, audience and aims of the speech in question. Then, I compare my findings with those of \citeauthor{Ye} and \citeauthor{Nur}.
}
\bibliographystyle{apalike}
\bibliography{B083350.bib}
\end{document}
