% !TEX options=--shell-escape

\documentclass[11pt]{article}

\usepackage{geometry}
\usepackage[hidelinks, colorlinks=true, linkcolor=blue, citecolor=blue]{hyperref}

% APA citation style
\usepackage{natbib}
\usepackage{apalike}

\usepackage{setspace} % line spacing

% Paragraph indentation (first line)
\setlength{\parindent}{0em}
% Paragraph spacing
\setlength{\parskip}{.7em}

% Custom horizontal rules in tables
% \usepackage{booktabs}

% multi-row table cells
% \usepackage{multirow}

% \usepackage{subfigure}
% \usepackage{subcaption}
\usepackage{caption}

% Names of subsections and figures in cross-references with \autoref{}
\renewcommand*{\subsectionautorefname}{Section}%
\renewcommand*{\figureautorefname}{Fig.}%

\usepackage{textcomp} % fancy symbols in text mode (e.g. right arrow)
\newcommand{\textapprox}{\raisebox{0.5ex}{\texttildelow}}

% For better enums, e.g. resuming after a normal line of text
% \usepackage{enumitem}

% For images
\usepackage{graphicx, rotating}

% Center caption text for images
\usepackage[justification=centering]{caption}

% \onehalfspacing
\title{\LARGE Introduction to Discourse Analysis -- Assignment 1}
\date{}

\begin{document}

\maketitle

\,
\vspace{-5em}

\section*{Introduction}{
	% - Nur & Ruijuan analyse the ways in which 2 successful political speeches achieve this
	% - in particular, they analyse the interpersonal metafunction (proposed by Halliday in context of SFL) of the language, focusing on mood modality and pronoun
	% - they observe that the most dominant tools employed are positive declaratives, modals with high degree of modal committment, plural 1st person pronouns
	% - they also notice less frequent but effective careful use of imperatives, 2nd & 3rd person pronouns
	% - to see how their findings generalise, I apply their approach to Obama's inauguration speech

	% - political speech is a tool and does this, this, that
	Political speeches are more than their content, they are tools used to influence listeners and/or diminish the opposition.
	% - a successful speech will bridge gaps, make audience identify with the message, believe and support the speaker
	A successful speech will bridge the gap between the speaker and the audience, making listeners identify with the message, believe and support the politician.
	% - IM is great for analysing these things
	To analyse speeches from this perspective, one can turn to Interpersonal Metafunction, one of the functions of language proposed by \citet{Halliday1970IM} as part of his functional view of language.
	% - Nur & Ruijuan analyse IM in political speeches, on Mandela's 1994 inauguration and Obama's 2009 victory speech
	\citet{Ye} and \citet{Nur} analyse Interpersonal Metafunction in Barack Obama's 2008 presidential election victory speech and in Nelson Mandela's 1994 inauguration speech, respectively. 
	% - they analyse the same elements of IM, and they come to very similar conclusions w.r.t use of and role of IM elements in the two speeches
	Having analysed the same elements of the metafunction, their conclusions regarding the use and role of the metafunction elements in the two speeches are comparable. The conclusions are also practically the same.
	% - I challenge this, hypothesising that actually differences between inauguration and victory speeches could surface in the use of IM
	I challenge this agreement and hypothesise that victory and inauguration speeches differ in ways that should surface in Interpersonal Metafunction.
	% - victory is for supporters who already support the person, while inauguration is for everyone and needs to unite while presenting plans
	While victory speeches address supporters of the elect and will likely reflect on the successful campaign and celebrate, inauguration speeches present the program of the upcoming presidency and are aimed at the entire nation, thus needing to unite all citizens and make them believe and support the set common goals.
	% - I chose Obama's inauguration so that as many properties as possible stay the same (speaker, time period, affiliation, etc) and only the occasion changes
	To verify my hypothesis, I analyse a speech that shares the time period and speaker with that analysed by \citeauthor{Ye}: the 2009 inauguration speech of Barack Obama. 
	% - I'll explore the same IM elements Ruijuan did and will compare with Ruijuan's findings
	Thus, and by analysing the same elements of Interpersonal Metafunction as \citeauthor{Ye}, I keep most variables fixed while varying the occasion, audience and aims of the speech in question. Then, in section TODO I elaborate on the findings of \citeauthor{Ye} and \citeauthor{Nur} and juxtapose them with mine.
}

\section*{Theoretical background}{
	% - Halliday developed SFL
	% - language is functionally organised, it's shaped by the functions that is serves
	In his works in Systemic Functional Linguistics, Michael Halliday views language as shaped by the functions it serves, as inherently functionally organised. The different functions are thus directly related to the different kinds of meaning language carries.
	% - Halliday describes 3 such metafunctions and argues that, subsequently, the meaning of a language can be explained from the corresponding 3 different perspectives (_strands of meaning_):
	The language functions (or metafunctions) \citet{Halliday1970IM} proposes, together with the corresponding \textit{strands of meaning}, are:
	\begin{enumerate}
	%   - **experiential** metafunction as use of language for talking about reality, human experience of the world, etc., constituting **ideational** meaning
		\item \textbf{experiential metafunction} -- as an ``observer's'' tool for describing reality, experience, ideas; leading to \textbf{ideational meaning},
	%   - **interpersonal** metafunction -- language as tool/medium for social interaction; giving rise to the **interpersonal** (social) meaning concerned with interactions, relationships, etc
		\item \textbf{interpersonal metafunction} -- as an ``actor's'' tool for taking a stance or role, interacting with and influencing others; creating the \textbf{interpersonal meaning},
	%   - **textual** metafunction -- language as a communicative tool, a message -- the surface form, constructed so as to fit well with the local context (preceding messages) as well as wider (e.g. social) context of the message, leading to the **textual** meaning about language as a message
		\item \textbf{textual} metafunction -- as a ``communicator's'' tool for making the previous two meanings surface in a coherent way, considerate of the wider social and situational context as well as the local context of surrounding text; leading to \textbf{textual meaning}.
	\end{enumerate}
	% - the 3 metafunctions are intertwined, but the interpersonal one is the most important one in political speeches as it describes how language can be employed as a tool to bring the speaker closer to their audience (lessen the _social distance_), establish a positive or close relationship, influence, persuade, inspire the audience, etc.	
	The different functions co-exist in language and are intertwined, but the interpersonal metafunction is what makes language a social instrument like in political speeches. A speaker can use elements of the metafunction to assume a role, change the \textit{social distance} between them and the audience, establish positive or close relationships, unite or divide, inspire and persuade.
}

\section*{Data}{
	% - january 2009, 2.5 months after victory
	Barack Obama's inauguration speech took place in January 2009, 2.5 months after his victory speech (analysed by \citeauthor{Ye}).
	% - context of global financial crisis and war in Iraq and environmental challenges
	The world was amidst the global financial crisis, the US were involved in a war in Iraq, and the idea of an environmental crisis was becoming clearer.
	% - focus on facing the difficulties (like ancestors did)
	Obama names all these clearly and, by recalling the difficulties past generations of Americans faced throughout history with perseverance and high spirit, 
	% - it aims to unite all Americans and present US as a world leader
	tries to motivate the audience to unite and fight again.
	% - compared to victory speech, inauguration speech focuses more on presenting the main topics and program of the new president+government
	Compared to the victory speech, Obama focuses much more on uniting all citizens and presenting the focus points of his upcoming presidency.

	% - I chose Obama's 2009 inauguration speech because I expect interesting similarities and differences with previous speeches:
	%   - at the basic level, it's a presidential speech (like the others)
	%   - as an inauguration speech, it could be close (=> better comparable) to Mandela's speech more than the victory speech analysed by Ruijuan
	%   - as a speech by Obama, and coming only 2.5 months after his victory speech, I expect it to be similar
	%   - on the other hand, differences with Obama2008 can be related to the different occasions (election victory vs inauguration), audiences and purposes:
	%     - while the victory speech was aimed primarily for Obama's supporters, the inaugural speech was widely televised and addressed to all Americans and even to other nations
}

\bibliographystyle{apalike}
\bibliography{B083350.bib}
\end{document}
