% !TEX options=--shell-escape

\documentclass[11pt]{article}

\usepackage{geometry}
\usepackage[hidelinks, colorlinks=true, linkcolor=blue, citecolor=blue]{hyperref}

% APA citation style
\usepackage{natbib}
\usepackage{apalike}

\usepackage{setspace} % line spacing

% Paragraph indentation (first line)
\setlength{\parindent}{0em}
% Paragraph spacing
\setlength{\parskip}{.7em}

% Custom horizontal rules in tables
% \usepackage{booktabs}

% multi-row table cells
% \usepackage{multirow}

% \usepackage{subfigure}
% \usepackage{subcaption}
\usepackage{caption}

% Names of subsections and figures in cross-references with \autoref{}
\renewcommand*{\sectionautorefname}{Section}%
\renewcommand*{\subsectionautorefname}{Section}%
\renewcommand*{\figureautorefname}{Fig.}%

\usepackage{textcomp} % fancy symbols in text mode (e.g. right arrow)
\newcommand{\textapprox}{\raisebox{0.5ex}{\texttildelow}}

% For better enums, e.g. resuming after a normal line of text
% \usepackage{enumitem}

% For images
\usepackage{graphicx, rotating}

% Center caption text for images
\usepackage[justification=centering]{caption}

% styling of quotes (use like: \Quote{text}{label}), including their counting and referencing with \qref{label}
\newcounter{quotecount}
\newcommand{\Quote}[2]{\vspace{0.4cm}\\ 
	\hspace*{1cm}\parbox{12cm}{\em #1 \refstepcounter{quotecount}\label{#2}}\hspace*{1cm}(\arabic{quotecount})\\[0.4cm]}
\newcommand{\qref}[1]{(\autoref{#1})}


% \onehalfspacing
\title{\LARGE Introduction to Discourse Analysis -- Assignment 1}
\date{}

\begin{document}

\maketitle

\,
\vspace{-5em}

%%%%%%%%%%%%%%%%%%%%%
% total words: 1839 %
%%%%%%%%%%%%%%%%%%%%%

% 196
\section{Introduction}{
	% - Nur & Ruijuan analyse the ways in which 2 successful political speeches achieve this
	% - in particular, they analyse the interpersonal metafunction (proposed by Halliday in context of SFL) of the language, focusing on mood modality and pronoun
	% - they observe that the most dominant tools employed are positive declaratives, modals with high degree of modal committment, plural 1st person pronouns
	% - they also notice less frequent but effective careful use of imperatives, 2nd & 3rd person pronouns
	% - to see how their findings generalise, I apply their approach to Obama's inauguration speech

	% - political speech is a tool and does this, this, that
	Political speeches go beyond content, they are tools used to influence listeners. 
	% - a successful speech will bridge gaps, make audience identify with the message, believe and support the speaker
	Successful speeches bridge the speaker-audience gap, make listeners identify with and support the politician.
	% - IM is great for analysing these things
	To analyse speeches from this perspective, one can turn to Interpersonal Metafunction, a function of language proposed by \citet{Halliday1970IM}.
	% - Nur & Ruijuan analyse IM in political speeches, on Mandela's 1994 inauguration and Obama's 2009 victory speech
	\citet{Ye} and \citet{Nur} analyse Interpersonal Metafunction in Barack Obama's 2008 presidential election victory speech and in Nelson Mandela's 1994 inauguration speech, respectively.
	% - they analyse the same elements of IM, and they come to very similar conclusions w.r.t use of and role of IM elements in the two speeches
	Their findings regarding the use and role of the metafunction elements are practically the same.
	% - I challenge this, hypothesising that actually differences between inauguration and victory speeches could surface in the use of IM
	I challenge this agreement and hypothesise that victory and inauguration speeches differ in ways that should surface in Interpersonal Metafunction.
	% - victory is for supporters who already support the person, while inauguration is for everyone and needs to unite while presenting plans
	While victory speeches typically address supporters, celebrate and reflect on campaigns, inauguration speeches present priorities of upcoming presidencies and aim at entire nation, trying to unite all citizens behind the proposed goals.
	% - I chose Obama's inauguration so that as many properties as possible stay the same (speaker, time period, affiliation, etc) and only the occasion changes
	To verify my hypothesis, I analyse the 2009 inauguration speech of Barack Obama, which shares the time period and speaker with that analysed by \citeauthor{Ye}. 
	% - I'll explore the same IM elements Ruijuan did and will compare with Ruijuan's findings
	Thus, and by analysing the same elements of Interpersonal Metafunction as Ye, I keep most variables fixed while varying the occasion, audience and aims of speech. After the analysis, I juxtapose my findings with those of \citeauthor{Ye} and \citeauthor{Nur}.
}

% 149
\section{Theoretical background}{
	% - Halliday developed SFL
	% - language is functionally organised, it's shaped by the functions that is serves
	In Systemic Functional Linguistics, Halliday views language as shaped by functions it serves -- functionally organised. The different functions then relate to different kinds of meaning language carries.
	% - Halliday describes 3 such metafunctions and argues that, subsequently, the meaning of a language can be explained from the corresponding 3 different perspectives (_strands of meaning_):
	The (meta)functions \citet{Halliday1970IM} proposes, with corresponding meaning types, are:
	\begin{enumerate}
	%   - **experiential** metafunction as use of language for talking about reality, human experience of the world, etc., constituting **ideational** meaning
		\item \textbf{experiential metafunction} -- as an ``observer's'' tool for describing reality, experience, ideas; leading to \textbf{ideational meaning},
	%   - **interpersonal** metafunction -- language as tool/medium for social interaction; giving rise to the **interpersonal** (social) meaning concerned with interactions, relationships, etc
		\item \textbf{interpersonal metafunction} -- as an ``actor's'' tool for taking a stance or role, interacting with and influencing others; creating \textbf{interpersonal meaning},
	%   - **textual** metafunction -- language as a communicative tool, a message -- the surface form, constructed so as to fit well with the local context (preceding messages) as well as wider (e.g. social) context of the message, leading to the **textual** meaning about language as a message
		\item \textbf{textual metafunction} -- as a ``communicator's'' tool for making the previous two meanings surface coherently, considerate of wider social and situational context and of local context of surrounding text; leading to \textbf{textual meaning}.
	\end{enumerate}
	% - the 3 metafunctions are intertwined, but the interpersonal one is the most important one in political speeches as it describes how language can be employed as a tool to bring the speaker closer to their audience (lessen the _social distance_), establish a positive or close relationship, influence, persuade, inspire the audience, etc.	
	The different functions co-exist in language, but the interpersonal metafunction is what makes language a social instrument like in political speeches. Speakers use elements of the metafunction to assume role, change the social distance between them and audiences, establish relationships, unite or divide, inspire and persuade.
}

% 111
\section{Data}{
	% - january 2009, 2.5 months after victory
	Obama's inauguration speech took place in January 2009 (2.5 months after the victory speech).
	% - context of global financial crisis and war in Iraq and environmental challenges
	America was facing the global financial crisis and war in Iraq, and an approaching environmental crisis was becoming clearer.
	% - focus on facing the difficulties (like ancestors did)
	Obama names these directly and, by recalling the difficulties past generations of Americans faced with perseverance and high spirit (i.e. common American values grounded in history), 
	% - it aims to unite all Americans and present US as a world leader
	motivates citizens to unite and fight again.
	% differences: occasion, audience, goals
	Addressed to all Americans and to other nations, the speech tried to win support for the upcoming presidency's challenging goals.
	% similarities
	The speaker stays the same person with the same affiliation as in the victory speech. Hence, I hypothesise that any observed differences relate mainly to the different situations.

	% - I chose Obama's 2009 inauguration speech because I expect interesting similarities and differences with previous speeches:
	%   - at the basic level, it's a presidential speech (like the others)
	%   - as an inauguration speech, it could be close (=> better comparable) to Mandela's speech more than the victory speech analysed by Ruijuan
	%   - as a speech by Obama, and coming only 2.5 months after his victory speech, I expect it to be similar
	%   - on the other hand, differences with Obama2008 can be related to the different occasions (election victory vs inauguration), audiences and purposes:
	%     - while the victory speech was aimed primarily for Obama's supporters, the inaugural speech was widely televised and addressed to all Americans and even to other nations
}

% 991
\section{Analysis}{
	% 19
	%  - I analyse the dimensions of IM explored by both Nur and Ruijuan, i.e. mood, modality and personal pronouns.
	For comparability with \citeauthor{Nur} and \citeauthor{Ye}, I analyse the same elements of Interpersonal Metafunction: mood, modality and personal pronouns.
 	
	% 300 - 9(table)
 	%  - Mood
 	\subsection{Mood}{
		%    - The concept of Mood relates to the very nature of clause as exchange
 		Language interaction can be viewed as speaker-initiated exchange of commodity: The speaker gives or demands, the commodity being information or goods/services.
		%    - provides basis for interactions between addresser and audience in the form of 4 basic speech functions as exchanges of 2 types of commodity: 
 		The mood system is then ``the expression of the speaker's choice of role in the communicative situation'' \citep{Halliday1970ModalityMood}, where choosing to give/demand information leads to statements/questions, and giving/demanding goods-and-services leads to offers/commands. While choosing commands easily widens the speaker-audience social gap, statements or offers can achieve the opposite.
		%    - the mood itself is in a clause represented by subject + finite (typically the subject-verb pair)
		% statement (giving information), question (demanding information), offer (giving goods & services), and command (demanding g&s)

		%    - declarative/imperative/interrogative
		%      - observed: 106/9/0: 92/8/0%
		%      - Ruijuan: 101/5/2: 94/5/1%
		%      - Nur: 36/5/0: 88/12/0%
		\begin{table}[h!tb]
	      \centering
			\begin{tabular}{l|ccc}
			         & declarative & imperative & interrogative \\
			  \hline
			  \hline
			  here   & 92\%        & 8\%        & 0\% \\
			  \hline
			  Ye     & 94\%        & 5\%        & 1\% \\
			  Nur    & 88\%        & 12\%       & 0\% \\
			\end{tabular}
	      \caption{Clause types.}
	      \label{tab:mood}
	    \end{table}

 		Grammatically, statements are realised as declarative clauses, questions as interrogatives, and commands as imperatives. \autoref{tab:mood} shows that Obama uses almost exclusively declaratives -- assumes the role of information giver. Thus, he effectively communicates values that unite all Americans:
		%    - declarative examples
		\Quote{On this day, we gather because we have chosen hope over fear, unity of purpose over conflict and discord.}{declarative1}
 		but also persuades and motivates:
 		\Quote{Today I say to you that the challenges we face are real.}{declarative2}
 		\Quote{Starting today, we must pick ourselves up, dust ourselves off, and begin again the work of remaking America.}{declarative3}
 		while still sounding strong and relatable, not confrontational or patronising.

		%    - imperatives addressed to "others" countries and potential enemies (3x), and to all Americans (the Let form, 3x)
		%    - imperative example:
 		The few imperatives serve two roles. They demonstrate strength and courage when confronting enemies:
 		\Quote{\normalfont To those leaders around the globe who seek to sow conflict, or blame their society's ills on the West,\em\  know that your people will judge you on what you can build, not what you destroy.}{imperative1}
 		In the remaining cases -- when addressing Americans -- imperatives are used not to command, but instead use ``let'' to emotionally encourage listeners to join the action:
 		\Quote{With hope and virtue, let us brave once more the icy currents, and endure what storms may come.}{imperative2}
		%    - similar to previous findings. more imperatives in inaugural vs victory speeches may be attributed to presenting program and encouraging people to join and act in accordance with it
		Altogether, declaratives and imperatives positively assert key messages, showing Obama as a strong, protective leader who invites all citizens to join him.
 	}

 	% 289 - 27(table)
 	%  - Modality
 	\subsection{Modality}{
 		Modality refers to intermediate polarities from the ``yes''--``no'' continuum.
		%    - covering the range of meanings between yes and no, modality refers to how speaker's position, (un)certainty, (in)ability and attitude surface in semantic meaning
		%    - one can go further and distinguish between:
 		The system enables:
 		\begin{itemize}
		%      - modalization as speaker's judgement of proposition in terms of probablity (possible -> certain) and usuality (sometimes->always)
 			\item speakers to assume positions towards a proposition, in terms of its probability and usuality (termed \textit{modalization})
		%      - modulation as expressing speaker's sense of obligation (allowed->required) and inclination (willing->determined)
 			\item describing levels of obligation and inclination (termed \textit{modulation})
 		\end{itemize}
		%      - but Halliday 1970 argues that these are "the same system in different functions", whereas modalization changes the meaning at the interpersonal level while modulation changes the content of the proposition at the ideational level
		%    - both can be realised in various forms, using e.g. modal verbs, adjectives, adverbs...
		%    - I limit myself to modal verbs, analysed in Nur and Ruijuan
 		While modulation happens at ideational (content) level and modalization at interpersonal level (speaker expressing their stance), they both surface primarily as modal verbal operators (``can'', ``must'', ...). \citet{Halliday1970ModalityMood} notes that the ideational part of modality is \textit{oriented towards} the interpersonal component and vice versa; modulation and modalization are ``the same system in different functions''.
		%                | will/won't (future/wish) | would | must | can/cannot | could | be able to | other | total
		%    - observed: | 19         (13/6)        | 1     | 8(6) | 12/6       | 0     | 0          | 8     | 54 might(2) shall(3) may(3)
		%    - Ruijuan:  | 18         (7/11)        | ?     | 4    | 22         | ?     | ?          | 11    | 55
		%    - Nur:      | 6          (?/?)         | ?     | 3    | 1          | 1     | 1          | 3     | 15
		\begin{table}[h!tb]
	      \centering
	        \begin{tabular}{l|cccccccc}
	                & will (future/wish) & would & must & can/cannot & could & be able to & other & total \\
	          \hline
	          \hline
	          here  & 35\%   (24\%/11\%) & 2\%   & 15\% & 22\%/11\%  & 0\%   & 0\%        & 15\%  & 54 \\
	          \hline
	          Ye    & 33\%   (13\%/20\%) & ?     & 7\%  & 40\%       & ?     & ?          & 20\%  & 55 \\
	          Nur   & 40\%   (?/?)       & ?     & 20\% & 7\%        & 7\%   & 7\%        & 20\%  & 15 \\
	        \end{tabular}
	      \caption{Modal verb operators.}
	      \label{tab:modality}
	    \end{table}

		%    - Most dominant is the high commitment (will, must), but note that of 19 wills only 6 communicate strong will/determination, while 13 are simply markers of future tense. also, I identify 2 cases of must which I don't see as communicating urge, but simply obligation:
	    \autoref{tab:modality} shows that the strongest modals (will, must) dominate the inauguration speech, but, looking carefully, only 1/3 of ``will''s express determination/strong wish:
	    \Quote{All this we can do.\\And all this we will do.}{will1}
	    while the remainder are ``will'' used merely as future tense marker in presenting Obama's plans:
	    \Quote{Where the answer is no, programs will end.}{will2}
	    Similarly with ``must'': In 6 cases it communicates urge and obligation like in \qref{declarative3}, but in 2 cases carries no strong charge:
	    \Quote{\normalfont [...] greatness is never a given.\\ \em It must be earned.}{must1}
	    Overall, the high commitment modals successfully convey Obama's determination in an emotional way, but also many of them don't contribute to the interpersonal metafunction.

		%    - can is also dominant, BUT used only in 2x cases to strongly communicate the ability (in all other cases it's either negative, or communicating in a rather ordinary way the possibility/ability) -- note that Obama here doesn't use 'yes we can':
	    The frequent ``can'' must also be analysed carefully. I find only 2 instances where ``can'' emotionally persuades audience about their abilities; see \qref{will1} and:
	    \Quote{Guided by these principles once more, we can meet those new threats that demand even greater effort [...]}{can1}
	    the remaining 16 instances used without much interpersonal effect to express ability/possibility, see \qref{imperative1} and:
	    \Quote{For as much as government can do and must do [...]}{can2}
	}

	% 240
	\subsection{Personal pronouns}{
		By choosing different pronouns, speaker can establish different speaker-audience relations; e.g. ``I'' emphasises the \textit{I vs you} gap, ``we'' can bridge this gap, and ``they'' emphasises the \textit{we vs they} gap. Thus, the use of pronouns is an important tool for a politician.

		%  - Personal pronoun (everything in percent)
		%    -          | I  | We (excl./incl.) | Us | Our | You | Your | S/he | They/Their/them/it | total
		%    - observed | 1  | 27 (4/23)        | 10 | 30  | 6   | 1    | 0.5  | 8/4/2/8            | 224
		%    - Ruijuan  | 19 | 31               | 8  | 17  | 10  | 3    | 10   | 2/?/?/?            | 131
		%    - Nur      | 1  | 34               | 11 | 26  | 3   | 0    | 0    | 1/9/3/13           | 80
		\begin{table}[h!tb]
	      \centering
	        \begin{tabular}{l|ccccccccc}
	                & I    & we (excl./incl.) & us   & our  & you  & your & s/he  & they/their/them/it & total \\
	          \hline
	          \hline
	          here  & 1\%  & 27\% (4\%/23\%)  & 10\% & 30\% & 6\%  & 1\%  & 0.5\% & 8\%/4\%/2\%/8\%    & 224 \\
	          \hline
	          Ye    & 19\% & 31\%             & 8\%  & 17\% & 10\% & 3\%  & 10\%  & 2\%/?/?/?          & 131 \\
	          Nur   & 1\%  & 34\%             & 11\% & 26\% & 3\%  & 0\%  & 0\%   & 1\%/9\%/3\%/13\%   & 80 \\
	        \end{tabular}
	      \caption{Distribution of personal pronouns (numbers from \citeauthor{Ye} and \citeauthor{Nur} for reference).}
	      \label{tab:pronouns}
	    \end{table}

		%    - 1st person dominates, exclusively plural: aiming to unite people and portray Obama as one of them (compare with victory speech where I is used often). we is found in 53% of all main clauses (compare with 38% in victory and 66% in Mandela -- this shows how inauguration is different from victory)
	   	\autoref{tab:pronouns} shows that as much as 67\% of all personal pronouns are 1st-person plural forms, used mainly to portray all Americans as united and Obama as one of them, see \qref{declarative1}, \qref{declarative2}, \qref{declarative3}, \qref{imperative2}, \qref{will1}, \qref{can1}. Only 1/6 of all uses of ``we'' employs the exclusive, dividing \textit{we all vs they} sense -- mainly when talking about enemies:
	   	\Quote{[...] for those who seek to advance their aims by inducing terror and slaughtering innocents, we say to you now that our spirit is stronger and cannot be broken; you cannot outlast us, and we will defeat you.}{we1}
		%    - 2nd person used almost exclusively to address (confront) 3rd parties such as foreign countries (including enemies), setting a strong (and somewhat patronising) tone
	   	This directly confrontational tone is also the main way in which Obama uses 2nd-person pronouns (in 14/17 cases) -- rather then opting for the more indirect confrontation using 3rd person. Only in 3 cases does he use ``you'' to address Americans: when using ``I-you'' to establish an intimate bond and express humbleness:
	   	\Quote{My fellow citizens: I stand here today humbled by the task before us, grateful for the trust you have bestowed [...]}{you1}
		%    - 3rd person (mainly they/their) is used to refer to those not present, mainly to earlier generations who worked/fought hard so the present-day Americans could live decent lives, and to American soldiers. By referring to these shared American values and symbols, Obama unites the audience.
		As for 3rd-person pronouns, Obama uses them to talk of those not present whom he doesn't want to confront in any way, rather to respect them: the past generations and the troops. Thus, he grounds his speech in these values widely shared by Americans:
		\Quote{For us, they fought and died, in places like Concord and Gettysburg; Normandy and Khe Sahn.}{they1}
	}
}

% 440
\section{Comparing with Nur and Ye}{
	\label{sec:comparison}
	% afaik, both explore how IM helps achieve aims of the speech
	Both \citeauthor{Nur} and \citeauthor{Ye} explore how Interpersonal Metafunction is used to make speeches succeed.
	% but the speeches have different aims (I think), and hence once can expect different uses of IM
	I reason that victory and inauguration speeches have different aims and hence one should expect different uses of the metafunction for success.
	% this isn't visible in the conclusions! but it should be because IM is what makes speeches good tols, right?
	However, looking at the conclusions of \citeauthor{Nur} and \citeauthor{Ye}, I see no principal differences.
	% so, if their findings are to be generalizable (in that we can continue to use IM to explain how successful speeches work), I wanna be able to also see differences
	Hence, I carry out a more detailed comparison and include results of my analysis. If I am to confirm generalizability of the results (and hence usefulness of Interpersonal Metafunction as a tool for analysing political speeches), I expect to find differences that are relatable to the different aims of the speeches.

	%    - **MOOD: previous findings**:
	%      - Nur: 
	%        - declarative clauses dominate {_making the speech solemn and persuasive, thrivingly recalling Mandela and his countrymen’s long sufferings, expressing gratitude to cowarriors and supporters and promising to work for the actual freedom_}
	%        - followed by imperatives (non-commanding Let) {_helping Mandela arouse people’s passion to dream and act together to earn equality & freedom; minimizes the social distance; making speech more emotive, appealing, and inspiring._}
	%      - Ruijuan:
	%        - declaratives dominate {_give as much as possible information to the audience, recalling election campaign, expressing gratitude to supporters, promising, inspiring the audience to face difficulties as one_}
	%        - imperatives (Let) follow {_shorten distance, call to act together & overcome the difficulties.; make speech more moving, appealing and inspiring_}
	
	% same as my findings, numbers agree, everything OK
	Regarding mood, all 3 sets of findings are very similar: Declaratives hugely dominate, communicating ideas and values, uniting speakers with listeners and inspiring. The (non-commanding) infrequent imperatives follow, bringing passion and emotion, encouraging people to act together and confronting enemies.

	%    - **MODALITY: previous findings**
	%      - Nur
	%        - will dominates (as future predicator & symbol of determination)
	%        - must follows {_conveys strong determination; calls audience to be determined to act towards common objectives_}
	%        - scarse can {_encourages people to believe in their abilities to act together; minimizes gap_}
	%      - Ruijuan
	%        - can dominates {_weaken authority & shorten distance. 'Be able' showed in repeated “Yes we can”, encourages people to believe in themselves & ability to climb back into the light_}
	%        - will follows {_shows strong mind and keen desire to lead through the difficulties; higher modal commitment confirms more actions will be definitely taken in the future_}
	%        - must follows {_shows firm determination to overcome difficulties, call to actions to achieve targets_}
	
	% different frequencies in must (but conclusions same!); I say that inauguration => more musting, also checks with Nur
	Regarding modality, I argue that \autoref{tab:modality} uncovers conflicting points. It shows higher use of ``must'' in inauguration speeches, yet the conclusions in all three cases are practically identical: \textit{that ``must'' shows determination and calls to action}.
	% I see that inauguration => more future will, less wishing will (cannot confirm with Nur...)
	I also see that \citeauthor{Ye}'s and my conclusions about the uses of ``will'' are very similar, yet the victory speech is dominated by ``will'' in the sense of strong determination while Obama's inauguration uses it mainly for describing future actions.
	The rest of the conclusions and statistics mostly agree well, even though the use of ``can'' in the victory speech is skewed by the 6x repeated \textit{yes we can}.
	% shows how repetition can skew numbers: yes we can repeated 6x
	Altogether, even where conclusions are similar, by comparing the statistics I see that inauguration speeches use ``must'' and the future ``will'' more frequently, which I relate to the aim of proposing and motivating plans for the presidency. 

	%    - **PRONOUNS: previous findings**
	%      - Nur
	%        - 1st: plural replaces singular {_inclusive kindles to feel same as Mandela, come forward & work together. exclusive implies authority & power. government integrated with people, grateful, with high spirit and powerful, ready to protect and unite._}
	%        - 2nd: least frequent {_crosses audience’s boundary, shows humble gratitude to international body_}
	%        - 3rd: frequent {_shows care/respect to people who died during the struggle, invites audience to feel same. Ties speaker+audience in intimate bondage_}
	%      - Ruijuan
	%        - 1st: most frequent, mainly plural, {_'I' makes Obama sincere person who remembers gratitude and will repay it. Inclusive 'we' makes Obama on the same boat as audience & persuades to share his proposals. Exclusive 'we' makes them strong leaders; wins Americans’ confidence in new government._}
	%        - 2nd: infrequent {_“you” attracts attention + creates dialogical feel; Obama shows care/respect to audience => intimate relation_}
	%        - 3rd: infrequent

	% MISMATCH in use of I, not visible from conclusions!!! shows importance of comparing numbers
	In use of personal pronouns, the statistics and findings are mostly similar, with a few important exceptions. 
	% - Ye: Obama uses I ``to speak of his election campaign and express his gratitude. '', `` application of “I” here successfully describes the new elected president into a sincere person who will remember the gratitude and try to repay it''
	The use of ``I'', which is much higher in the victory speech, even though the conclusions about its use are similar in both of Obama's speeches (\citeauthor{Ye} concludes that ``I'' portrays Obama as a sincere person ready to repay his gratitude).
	% - Ye also has more ``you'' which is connected to using ``I and you'', not we
	The higher use of ``you'' in the victory speech can be related to that of ``I'' in terms of use of the ``I-you'' pattern instead of ``we'' like in Obama's inauguration, which I relate to the victory speech focusing much more on personal reflection on the campaign of the elect, while the inauguration is much more about the common journey ahead.
	% - s/he shows how small things can skew stats: in this case repeating things about Ann Nixon Cooper (most of these pronouns refer to her)
	Finally, another example of statistics skewed by unpredictable phenomena is visible in the frequent use of ``she'' in the victory speech, caused mostly by the use of (and repeated referring to) an anecdote involving Ann Nixon Cooper.
}

% 131
\section{Conclusions}{
	% mostly agreeing conclusions as Nur and Ye
	My analysis shows results which mostly agree with those of \citeauthor{Ye} and \citeauthor{Nur}.
	% point out differences between Nur+Ye, and also with my analysis, and explain and relate to differences in inauguration vs victory aims
	The differences found in the use of personal pronouns and modals I am able to relate to the different aims of victory vs inauguration speeches. Hence, I demonstrate that analysis of Interpersonal Metafunction is useful here -- which would not hold if it led to the same conclusions for clearly different speeches.
	% thus, I confirm generalizability of Nur and Ye's results in that my results agree and IM can explain and account for differences in aims
	Thus, I confirm generalizability of \citeauthor{Ye} and \citeauthor{Nur}'s results in that Interpersonal Metafunction sheds light into structures behind successful speeches, while similarities and differences in results reflect the differences stemming form the situational and/or social context of speeches.
	% I also warn that conclusions should reflect stats unless there is suspicion that stats are skewed
	Additionally, I advocate for stronger presence of raw statistics in formulating findings. While numbers alone can be skewed due to uncontrollable phenomena,  verbal conclusions alone can hide important statistical differences.
}

\bibliographystyle{apalike}
\bibliography{B083350.bib}

\section*{Appendix}{

}
\end{document}
