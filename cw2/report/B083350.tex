% !TEX options=--shell-escape

\documentclass[11pt]{article}

\usepackage{geometry}
\usepackage[hidelinks, colorlinks=true, linkcolor=blue, citecolor=blue]{hyperref}

% APA citation style
\usepackage{natbib}
\usepackage{apalike}

\usepackage{setspace} % line spacing

\usepackage{bm}

% Paragraph indentation (first line)
\setlength{\parindent}{0em}
% Paragraph spacing
\setlength{\parskip}{.7em}

\usepackage{caption}

% Names of subsections and figures in cross-references with \autoref{}
\renewcommand*{\sectionautorefname}{Section}%
\renewcommand*{\subsectionautorefname}{Section}%
\renewcommand*{\figureautorefname}{Fig.}%

\usepackage{textcomp} % fancy symbols in text mode (e.g. right arrow)
\newcommand{\textapprox}{\raisebox{0.5ex}{\texttildelow}}

% For images
\usepackage{graphicx, rotating}

% Center caption text for images
\usepackage[justification=centering]{caption}

% styling of quotes (use like: \Quote{text}{label}), including their counting and referencing with \qref{label}
\newcounter{examplecount}
\newcommand*{\examplecountautorefname}{E\kern-0.3em}
\newcommand{\Example}[3]{
	\vspace{0.4cm}\\
	\refstepcounter{examplecount}
	\hspace*{1cm}\parbox{12cm}{(E\arabic{examplecount}) #1}

	\hspace*{1.5cm}\parbox{12cm}{\texttt{#2}\label{#3}}\\[0.4cm]
}
\newcommand{\eref}[1]{(\autoref{#1})}

\usepackage{listings}
\lstset{
  columns=flexible,
  basicstyle=\ttfamily,
  mathescape=true,
  escapeinside=||
}

% \onehalfspacing
\title{\LARGE Introduction to Discourse Analysis -- Assignment 2 \\ Task 5}
\date{}
\author{}

\begin{document}

\maketitle

\,
\vspace{-5em}

%%%%%%%%%%%%%%%%%%%%%
% total words: XXXX %
%%%%%%%%%%%%%%%%%%%%%

% INSTRUCTIONS
% In completing the tasks, you are expected to demonstrate your understanding of relevant theoretical concepts and frameworks (review section), your ability to apply that understanding to a specific case/situation and relevant academic writing skills. 
% In the theoretical section, feel free to draw on data from the literature to illustrate the various theoretical points. 
% Likewise, in the analysis section, feel free to draw on examples from the literature to support further your claims.

% TASK 5
% Marcia’s turn in lines 21-23 in the extract below appears to be an account. 
% Yet, it is not obvious what it is an account of. 
% Describe the sequential organisation of talk in the extract and show how the account comes about.
% The question is about how the speaker ends up having to utter an account.

% terms: hedge, false start, warrant, dispreferred FPP

% TRANSCRIPTIONS:
% - .hhh: inbreath
% - hhh: outbreath
% - =: continuous transition from one TCU to another
% - {}
% - []: overlapping talk
% - (.): brief interval
% - word-: cut-off

\section{Introduction}{

}

\section{Theoretical background}{
	% - CA (fathers Schegloff, Sacks) about social actions in speaking, not about speaking itself
	Within Conversation analysis as focused on \textit{actions} in talk, not just on speaking itself,
	% 	- actions broader than talking: can be non-verbal too
	% - speaking is mostly sequential; seq. org. about how actions are ordered in interaction (conversation), how an action influences what comes next, eventually constructing orderly sequences of talk
	Sequantial organisation explores how actions (requesting, inviting, telling...) are ordered when they surface in interaction (conversation); how an action influences successive actions so that orderly sequences of talk are created.
    
    % - notion of ``conditional relevance'': an action makes a certain next action(s) relevant (or even expected), e.g. a question makes an answer be the relevant next action.
    The concept of \textit{conditional relevance} \citep{Schegloff_1968} describes how one action makes certain next action(s) relevant (suitable, or even or required if the talk is to proceed orderly), e.g. a question inviting an answer.
    % - hence adjacency pairs: basic units of seq. org. \citep{Schegloff_1973}. defined by \citet[p.~64]{Sidnell_2010} as: ``Adjacent. Produced by different speakers. Ordered as a first pair part (FPP) and a second pair part (SPP). Typed, so that a particular FPP provides for the relevance of a particular SPP (or some delimited range of SPPs)''
    Thus, long action sequences consist of basic action pairs -- \textit{adjacency pairs}; defined by \citet[p.~64]{Sidnell_2010} as:
    \begin{enumerate}
    	\item Adjacent
    	\item Produced by different speakers
    	\item Ordered as a first pair part (FPP) and a second pair part (SPP)
    	\item Typed, so that a particular FPP provides for the relevance of [occasions] a particular SPP (or some delimited range of SPPs)
    \end{enumerate}
    The type of a pair can be e.g. greeting-greeting or question-answer, an example adjacency pair of the latter type being:
    % - example adj. pair \Example{from \citet[p.~107]{Liddicoat_2007}}{John: What time' s it? \\Betty: Three uh clock.}{adj-pair}
    \Example{from \citet[p.~107]{Liddicoat_2007}}{John: What time' s it? \\Betty: Three uh clock.}{adj-pair}
    %   - the two lines are FPP and SPP, adjacent, produced by different speakers, and <<typed>> -- FPP is a question, which makes relevant a SPP of type answer
    %   - also note that speaker 1 gives speaker 2 space to react (in this case they stop speaking), or else they can't expect a SPP to occur!
    Notice too how John stops speaking to give Betty space to produce a SPP (or else there would be no interaction!)
    %   - also, a FPP requires a relevant SPP! if Betty stayed silent, John would likely repeat his question (unless he'd interpret the silence a relevant response)
    and how Betty produces a relevant SPP -- if she stayed silent, John could repeat his question (a FPP requiring a SPP) or interpret Betty's silence as a response.
    % - by producing a matching SPP, speaker 2 shows they understood the FPP, and a conversation continues orderly
    By producing a matching SPP, Betty also shows understanding of the preceding FPP and the conversation continues smoothly.

    % - a FPP doesn't occasion one particular SPP (and a meaningful reaction can also be non-verbal, e.g. silence or in \eref{adj-pair} Betty showing John her wristwatch)
    Importantly, an FPP can occasion multiple SPPs (including non-verbal actions, e.g. Betty showing John her wristwatch),
    % - BUT given a FPP, not all relevant SPPs may be equally valued: ``some responses are problematic for social relationships, while others are not'' \citep[p.~111]{Liddicoat_2007}
    but they are often unequally valued -- ``some responses are problematic for social relationships, while others are not'' \citep[p.~111]{Liddicoat_2007}.
    % - we can talk of usual/typical reactions as preferred (e.g. Betty's answer in \eref{adj-pair}), whereas less normal/socially appropriate reactions are termed dispreferred (e.g. Betty answering ``I don't know'' instead).
    Usual/unsurprising reactions are termed \textit{preferred} (e.g. Betty's particular answer in \eref{adj-pair}); the less socially appropriate/normal are \textit{dispreferred} (e.g. if Betty answered ``I don't know'').
    % - typically, the participants are aware of the social norms which make certain reactions preferred or dispreferred, and make the two surface differently in their speaking:
    Aware of social norms, participants typically treat the two types differently in speaking:
    % - preferred reactions, such as accepting an invitation, typically surface as short and immediate, like A's accepting in \eref{pref}
    preferred reactions (e.g. accepting an invitation) surface as short and immediate:
    \Example{from \citet[p.~58]{Casson_1981}}{B: Why don't you come up and see me some[times\\A: \hspace{18.5em}[I would like to}{pref}
    % - dispreferred reactions, on the other hand, surface in ways that mitigate their potential negative social impact, see 
    while dispreferreds surface in ways that mitigate their potential negative social impact:
    \Example{adapted from \citet[p.~110]{Liddicoat_2007}}{Harry: I don' have much tuh do on We:nsday.\\\phantom{.}~\hspace{2.9em}(.)\\\phantom{.}~\hspace{2.9em}w' d yuh like tuh get together then.\\\phantom{.}~\hspace{2.9em}(0.3)\\Joy: \ \ huh we::llhh I don' really know if yuh see\\\phantom{.}~\hspace{2.9em}i' s a bit hectic fuh me We:nsday yih know}{dispref}
    % - the rejection of the invitation is nearly not as overt as the acceptance. it comes with a delay of 0.3s, the hesitant ``we::ll'' with out-breathing, with hedges (``I don' really know'' and ``\textbf{a bit} hectic''), and explanations (accounts) for the refusal rather than a clear refusal itself. these signs are termed dispreferrence markers -- compare with shortness, overtness and lack of account as possible preferrence markers.
    Joy's rejection is not overt at all, comes with 0.3s delay, a hesitant ``we::ll'', hedges (``I don' really know'' and ``\textbf{a bit} hectic''), and explanations (accounts) rather than a clear refusal itself. These common signs -- \textit{dispreference markers} -- contrast with the shortness, overtness and lack of account (\textit{preference markers}) in \eref{pref}.
    
    % - as \citet{Liddicoat_2007} points out, not just SPPs, but also FPPs can be dispreferred, e.g. requests, which are then ``held back as later topics'' and ``accompanied by accounts and mitigations, which occur before the request itself'' -- i.e. dispreferrence markers, as seen in XXX, following Jim's previous (preparatory) explanations of his pitiful situation:  he provides a lengthy account and then makes the actual request in an indirect way using ``was wondering'' and ``could''.
    Importantly, FPPs can be dispreferred too (e.g. requests) and are then ``held back as later topics'' and ``accompanied by accounts and mitigations, which occur before the request itself'' \citep[p.~122]{Liddicoat_2007}:
    % - \Example{from \citet[p.~122]{Liddicoat_2007}}{Jim: well my car has broken down an they don' know\\\phantom{.}~\hspace{2.1em}if it will be fixed by then an' I w' z wondering\\\phantom{.}~\hspace{2.1em}if I c' d borrow your car.}{dispref-fpp}
    \Example{from \citet[p.~122]{Liddicoat_2007}}{Jim: well my car has broken down an they don' know\\\phantom{.}~\hspace{2.1em}if it will be fixed by then an' I w' z wondering\\\phantom{.}~\hspace{2.1em}if I c' d borrow your car.}{dispref-fpp}
    Following Jim's previous (preparatory) explanations of his pitiful situation, first 2 lines provide a lengthy account, before making the actual request in an indirect way using ``was wondering'' and ``could''.

    pairs can be expanded to create longer sequences: pre-, post- and inserts. 
    w.r.t. preference, a dispreferred tends FPP e.g. the main request sequence tends to be preceded by a pre-sequence which checks if preconditions are met (e.g. if the person needs their car atm), and which may make the request more relevant (and less dispreferred) by perhaps explaining the reasons for it.
}

\section{Analysis}{
	At a high level, the data (shown in \autoref{fig:data-marked}) is a short call, whereby Donny calls Marcia and starts describing how his car broke down in Glen while he needs to be in Brentwood -- supposedly fairly soon, and he supposedly needs a new car to get there. Marcia interrupts him and explains how she would love to help, but she needs to leave for some duties very soon (supposedly using her car). Then, Donny closes the talk very quickly, planning to call someone else (supposedly to borrow a car from them).

	\begin{figure}[h!tb]
		\centering
		\includegraphics[width=\columnwidth]{../data-marked.pdf}
		\caption{The data with sequence organisation graphically added by me on the left. Black circles mark FPPs, blank circles SPPs; $\boldsymbol{\mathbf{\otimes}}$ marks minimal post-expansion and $\boldsymbol{\mathbf{\oplus}}$ non-minimal post-expansion. Where connected by lines without arrowheads (lines 3, 5), and similarly where curly braces are used (e.g. lines 21--23), it means that multiple lines of the transcription form one pair part.}
		\label{fig:data-marked}
	\end{figure}

	I will now chronologically step through and describe the excerpt's sequential organisation, particularly with respect to Marcia's account (L21-23). (All adjacency pairs I also marked directly in \autoref{fig:data-marked}.)

	L1-2 summon-answer pair, notice non-verbal action L1
	L2 as FPP inquires who the caller is, upon which Donny identifies himself (L3+5).
	Marcia's interruption (L4) is not clear, but could signal that she recognised Donny already from L3, or that she confirms her identity after Donny addressed her by name in L3. Either way, Donny was not finished and ends his FPP in L5.
	L3+5 with L6 is a classical greeting-greeting pair which ends the opening section of the call. Note that this is a short form; a common, longer one would also add ``how are you?'' to the greetings, potentially lengthening the call opening with further chit-chat.

	L7-8 is a pre-sequence, a typical pre-telling pair: Donny signals he has something to tell and Marcia gives him a ``go-ahead'' signal. Then, the telling itself runs from L9-L20 (which will later turn out to be a long pre-sequence). Donny's FPP (L9) announces unfortunate news, making perhaps a sympathetic SPP relevant, but silence follows (L10), attributable to Marcia's lack of relevant reaction. Thus Donny makes a post-expansion (L11) by introducing further bad news, making a sympathetic reaction from Marcia even more relevant (if not required!). To L11 as a FPP, Marcia finally responds (L12-13), but her ``Oh::'' likely merely signals her receival of the new information from Donny\footnote{see detailed analysis of the role of ``oh'' \citep{Heritage_1984}}, but she goes silent (L13) instead of taking a stance or assessing Donny's situation. Donny accepts this and definitely closes the sequence by adding a minimal post-expansion (L14) -- merely an outbreath -- so that he can move on to his point.

	Across L15-20, Donny starts the main sequence (for which his telling in L7-14 was a pre-sequence), and puts forward his request as one very long FPP. The dispreferredness is realised using multiple common markers: hesitations (L15, then breathing in L17, and tokens ``a:t u:h'' in L20), pauses (L16, 17, 19, 20), only slowly moving to the point itself, placing it after a hedge (``I don't know if it's:ssible'') and a warrant (``see I haveta open up the ba:nk''). Without directly asking Marcia for her car, the request comes across very clearly and Marcia reacts, producing the SPP in L21-23.

	Marcia's reaction is also dispreferred and heavily marked: containing short pauses and hesitations where she stops (see the \verb|word-| pattern) to repairs her talk. Starting with a sympathetic ``Yeah:'', acknowledging Donny's pitiful situation, she then confirms she understands the request (``en I know you want-'') and shows her willingness to help (``en I would''), before finally providing an account for her inability to help right now (``I've gotta leave in aybout five min(h)tes'').

	Without ever directly refusing the request, Marcia's SPP is understood by Danny; he post-expands the sequence (L24-25), acknowledges the refusal (``Okay then'') and lessens its dispreferredness by outlining his contingency plans (to call someone else). The action also becomes a FPP which, being a summary of his situation, starts a typical \textit{sequence-closing sequence}. Such sequence, as discussed by \citet[p.~168]{Liddicoat_2007}, consists of 1) a summary/assessment proposing to close, 2) a go-ahead signal for closing, 3) the final turn, closing the sequence. However, Marcia doesn't co-operate -- stays silent (L26), which makes Donny post-expand (L27), repeating his suggestion to close more explicitly (``okay?''). To his post-expansion as a new FPP, Marcia responds quickly with the preferred SPP (L28), agreeing to close the sequence, after which Donny readily ends the conversation with a typical closing pair (L29-30).

	Looking back at the sequences in \autoref{fig:data-marked}, besides the very short opening and closing ones, the conversation consists of: a pre-telling (a pre-pre-sequence, L7-8), a pre-request pre-sequence (L9-14), and the main request sequence with a dispreferred SPP and a lengthy closing (L15-28).
}

\section{Conclusions}{

}

\bibliographystyle{apalike}
\bibliography{B083350.bib}

\section*{Appendix -- the raw data}{
\textbf{Marcia and Donny, stalled}
\begin{lstlisting}
1           1+ ring
2 Marcia:   Hello?
3 Donny:    'lo Marcia,=
4 Marcia:   Yea[:h  ]
5 Donny:      =[('t's) D]onny.
6 Marcia:   Hi Donny.
7 Donny:    Guess what.hh
8 Marcia:   What.
9 Donny:    hh My ca:r is sta::lled.
10 (0.2)
11 Donny:   ('n) I'm up here in the Glen?
12 Marcia:  Oh::
13 (0.4)
14 Donny:   {hhh}
15 Donny:   A:nd.hh
16 (0.2)
17 Donny:   I don't know if it's:ssible,  but {hhh}/(0.2)}see
18  I haveta open up the ba:nk.hh
19 (o.3)
20 Donny:   a:t uh: (.) in Brentwood?hh=
21 Marcia:=Yeah:- en I know you want-(.)En I whoa- (.) en I
22  would, but- except I've gotta leave in aybout five
23  min(h)tes. [(hheh)
24 Donny:       [Okay then I gotta call somebody
25 else. right away.
26 (.)
27 Donny:   okay?=
28 Marcia:  =okay [Don   ]
29 Donny:       [Thanks a lot. =Bye-
30 Marcia:  Bye
\end{lstlisting}
}

\end{document}
