% !TEX options=--shell-escape

\documentclass[11pt]{article}

\usepackage{geometry}
\usepackage[hidelinks, colorlinks=true, linkcolor=blue, citecolor=blue]{hyperref}

% APA citation style
\usepackage{natbib}
\usepackage{apalike}

\usepackage{setspace} % line spacing

% Paragraph indentation (first line)
\setlength{\parindent}{0em}
% Paragraph spacing
\setlength{\parskip}{.7em}

\usepackage{caption}

% Names of subsections and figures in cross-references with \autoref{}
\renewcommand*{\sectionautorefname}{Section}%
\renewcommand*{\subsectionautorefname}{Section}%
\renewcommand*{\figureautorefname}{Fig.}%

\usepackage{textcomp} % fancy symbols in text mode (e.g. right arrow)
\newcommand{\textapprox}{\raisebox{0.5ex}{\texttildelow}}

% For images
\usepackage{graphicx, rotating}

% Center caption text for images
\usepackage[justification=centering]{caption}

% styling of quotes (use like: \Quote{text}{label}), including their counting and referencing with \qref{label}
\newcounter{quotecount}
\newcommand{\Quote}[2]{\vspace{0.4cm}\\ 
	\hspace*{1cm}\parbox{12cm}{\em #1 \refstepcounter{quotecount}\label{#2}}\hspace*{1cm}(\arabic{quotecount})\\[0.4cm]}
\newcommand{\qref}[1]{(\autoref{#1})}

\usepackage{listings}
\lstset{
  columns=flexible,
  basicstyle=\ttfamily,
  mathescape=true,
  escapeinside=||
}

% \onehalfspacing
\title{\LARGE Introduction to Discourse Analysis -- Assignment 2 \\ Task 5}
\date{}

\begin{document}

\maketitle

\,
\vspace{-5em}

%%%%%%%%%%%%%%%%%%%%%
% total words: XXXX %
%%%%%%%%%%%%%%%%%%%%%

% reading plan
% 1. Sidnell (Action and Understanding)
% 2. Liddicoat (Adjacency Pairs and Preference Organisation)
% 3. Sidnell (Preference)

Sidnell (Action and Understanding)
- A great deal of talk is organized into sequences of paired actions or “adjacency pairs”.
- By the conditional relevance of one item on another we mean: given the first, the second is expectable; upon its occurrence it can be seen to be a second item to the first; upon its non-occurrence it can be seen to be officially absent – all this provided by the occurrence of the first item. Schegloff (1968: 1083)
- the relationship between paired utterance types such as question and answer is a norm to which participants themselves orient in finding and constructing orderly sequences of talk
- question and answer, request and granting, offer and acceptance, greeting and greeting, complaint and remedy
- Adjacent. Produced by different speakers. Ordered as a first pair part (FPP) and a second pair part (SPP). Typed, so that a particular first pair part provides for the relevance of a particular second pair part (or some delimited range of seconds)
- To compose an adjacency pair, the FPP and SPP come from the same pair type. (Schegloff, 2007)
- a failure to answer prompts the pursuit of a response
- first pair part of an adjacency pair has the capacity to make some particular types of conduct noticeably or relevantly absent, so that their non-occurrence is just as much an event as their occurrence.
- if a question is not answered, the questioner is likely to draw the inference that the recipient does not know the answer (or has some other reason for not answering
- a second pair part displays its speaker’s understanding of the first to which it responds.
- participants in conversation look to a next turn to see if and how they have been understood.
- Conversation analysis, on the other hand, examines the relation between practices of speaking and actions-in-talk within sequences.
- there’s typically no one-to-one mapping between some particular practice of speaking and some particular action

Liddicoat (Adjacency Pairs and Preference Organisation)
- turns at talk are places in which the participants in a conversation perform actions through talk
- in the sentence ``Could you open the window?'' it is problematic to consider this as primarily an utterance about windows and much more useful to consider it as enacting a request to do something.
- some actions make other actions relevant as next actions
- <<Schegloff and Sacks (1973) are the fathers of the term ``adjacency pairs''?? maybe!>>
- (1) consist of two turns (2) by different speakers, (3) which are placed next to each other in their basic minimal form, (4) which are ordered and (5) which are differentiated into pair types.
- when an FPP initiates a sequence, not just any SPP can occur in the second position: the SPP must be of the appropriate type for the action initiated by the FPP
- a question must be followed by an answer to be heard as a completed sequence: it cannot be followed by a greeting or a farewell
- <<some GOOD examples of adjacency pairs on p. 107>>
- These two turns together accomplish an action.
- once a recognizable FPP has been produced, on the first possible completion, the current speaker should stop and a next speaker should start and produce an SPP of the
relevant type
- FPPs create a context in which some next action is expected to occur
- Talk which is not an SPP does not remove the relevancy of an SPP being done in following talk.
- <<Counters change the flow/trajectory of sequence; can cancel the relevance of the FPP>>
- A small number of adjacency pair types have only a single type of SPP. The most common are greeting adjacency pairs (hello, hi, etc.) and terminal adjacency pairs
- <<(Atkinson and Heritage, 1984) introduced ``preference''>>
- The concept of preference deals with the possible ways in which some conversational action may be accomplished.
- Actions which are routinely performed immediately, and whose immediate production is unremarkable are termed preferred actions, while those which would not normally be performed in this way are called dispreferred actions
- some responses are problematic for social relationships, while others are not
- If a speaker needs to produce a next turn which is dispreferred, then s/he needs to design the turn in different ways in order to do extra conversational work.
- question ``That was a great film, wasn't it?'' is designed is such a way as to project a yes response, while a slightly different design ``The film wasn't very good, was it?'' projects a no response
- Answers to either of these questions would be designed with reference to the preferences established by the design of the question turn.
- there is an overwhelming preference for answers to agree with the trajectories of the questions to which they respond
- An interesting result of the preference for contiguity is that if there are two questions in a turn, the last question usually gets answered first (Sacks 1987).
- preferred SPPs come early in their turns and are contiguous with the FPP, and dispreferred SPPs are delayed in their turns and are thus not contiguous with their FPPs
- << examples of delay before dispreferred second, starting on p. 113 >>
- reformulation removes the need for a dispreferred action and makes a preferred action as a possible next action
- <<dispreferred action preceded by: silence, tokens like uhm, uh, well; breathing, hedges such as ``I dunno'', warrants (providing an account) >>
- Preferred responses do not need to be explained, dispreferred responses do. The placement of the warrant before the dispreferred SPP serves a dual function: it defers
the SPP until later in the turn and also provides a context in which the SPP can be heard.
- Mitigation seems to be an important part of the construction of dispreferreds (Pomerantz, 1984), and dispreferreds are regularly produced as weak disagreements or may be withheld altogether.
- << speakers can delay dispreferreds within a single turn, OR withing the sequence! >>
- summary of preference organisation:
	- a preferred action is routinely performed without delay;
	- a dispreferred action is routinely delayed in its turn;
	- a dispreferred action is routinely prefaced or qualified in its turn;
	- a dispreferred action is routinely accomplished in a mitigated or indirect form;
	- a dispreferred action is routinely accounted for.
- << to invitation questions (yes-like designed), yes is strong and immediate, no isn't overt, with apology >>
- projected alternative arrangements are often found with refused invitations
- << in assessments, we can have upgraded agreement, same agreement, or weaker agreement (the latter two preferred less) >>
- [in assessments] Weaker agreements can preface disagreements
- details of preference organization are not a fixed feature of conversations, but rather it is context sensitive
  - i.e. disagreement can be the preferred thing if it's e.g. a response to critical self-assessment!
- Request FPPs are also routinely designed as dispreferred turns. They are often delayed in their conversation: that is, they may be held back as later topics, even when they are the prime reason for a conversation taking place. Also, requests are regularly accompanied by accounts and mitigations, which occur before the request itself and which delay the request in its turn. [+ see example p.122]
  - The dispreferred request FPP involves the one participant making an imposition on the other, while the preferred FPP involves a participant undertaking to do something of his/her own accord

Liddicoat (Expanding sequences)
- expansions can be sequences, then we can talk of ``pre-sequences'', etc.
pre-expansion
- hearable by participants as preludes to some other action
- Pre-expansions, particularly type-specific pre-expansions, have as a primary role avoidance of problems in talk, particularly in dealing with problematic responses to FPPs
- generic pre-sequences, which are used with any form of following talk, and type-specific pre-sequences, which are designed to lead to some particular kind of base sequence
  - generic pre-sequence: summons-answer (designed to gain the attention of a recipient)
    - The summons-answer does not achieve completion in and of itself, but rather makes a next action relevant as the result of its completion. In other words, it projects some future action as the reason for the summons-answer sequence and it is heard as prior to this action.
  - FPP of type-specific pre-sequences projects a particular next activity as relevant for talk [pre-invitation, pre-request, pre-offers...]
  - PRE-REQUESTS (p. 132)
    - blocking response which indicates that the preconditions for the request cannot be met
    - pre-requests can project a request so strongly that in many cases the pre-sequence itself can achieve the request
  - PRE-TELLINGS (pre-announcements) (p. 137)
    - The minimal form of a pre-telling is 'Guess what'
    - common function of pre-tellings is to alert the recipient that what is to follow is a telling of some news
    - Often, the response to a pre-telling is a go-ahead response
    - By prompting a guess through a pre-telling, the need to perform the dispreferred turn [such as announcing bad news] may be removed.
  - MULTIPLE PRE-EXPANSIONS p.141
    - expansion sequences are sequences, they too include points at which further expansion can be made
    - 
- << preliminaries to preliminaries (pre-pres): quite generic, can lead to type-specific pre-sequences p.139 >>
  - Jim's request to borrow a car involves relevant information which his recipient cannot be presumed to know (his car has broken down and will take time to repair) or cannot be presumed to recognize as relevant (he has a distant meeting)
insertion
- expansion which occurs within the adjacency pair itself and separates the FPP from the SPP
- Typically, insert expansion is launched by an FPP produced by the second speaker which requires an SPP for completion.
- relevance of an SPP in response to the FPP is maintained and the SPP is delayed by the insert, not cancelled by it
- Post-first insert expansion
  - address issues arising from the FPP
  - insert sequences can be implicated in the possible production of a dispreferred SPP
- Pre-second insert expansion
  - pre-second insert expansion orients to the SPP which has been made relevant, rather than to the preceding FPP
  - very common in service encounters [p.148]
  - insert sequences can also be used to determine whether the SPP will be preferred or dispreferred [p.149]
post-expansion
- possible for talk to occur after the SPP which is recognizably associated with the preceding sequence
- minimal: sequence-closing thirds 
  - Minimal post-expansion, unlike the other examples of expansion discussed so far, is not in itself a sequence.
  - oh [p.152], okay [p.154], assessments [e.g. to close talk opening and move on] [p.155], composites [p.157]
- non-minimal
  - 

% TASK 5
% Marcia’s turn in lines 21-23 in the extract below appears to be an account. 
% Yet, it is not obvious what it is an account of. 
% Describe the sequential organisation of talk in the extract and show how the account comes about.

% terms: hedge, false start, warrant, dispreferred FPP

\section{Introduction}{
}

\section{Theoretical background}{
}

\section{Conclusions}{
}

\bibliographystyle{apalike}
\bibliography{B083350.bib}

\section*{Appendix}{
\textbf{Marcia and Donny, stalled}
\begin{lstlisting}
1           1+ ring
2 Marcia:   Hello?
3 Donny:    'lo Marcia,=
4 Marcia:   Yea[:h  ]
5 Donny:      =[('t's) D]onny.
6 Marcia:   Hi Donny.
7 Donny:    Guess what.hh
8 Marcia:   What.
9 Donny:    hh My ca:r is sta::lled.
10 (0.2)
11 Donny:   ('n) I'm up here in the Glen?
12 Marcia:  Oh::
13 (0.4)
14 Donny:   {hhh}
15 Donny:   A:nd.hh
16 (0.2)
17 Donny:   I don't know if it's:ssible,  but {hhh}/(0.2)}see
18  I haveta open up the ba:nk.hh
19 (o.3)
20 Donny:   a:t uh: (.) in Brentwood?hh=
21 Marcia:=Yeah:- en I know you want-(.)En I whoa- (.) en I
22  would, but- except I've gotta leave in aybout five
23  min(h)tes. [(hheh)
24 Donny:       [Okay then I gotta call somebody
25 else. right away.
26 (.)
27 Donny:   okay?=
28 Marcia:  =okay [Don   ]
29 Donny:       [Thanks a lot. =Bye-
30 Marcia:  Bye
\end{lstlisting}
}

\end{document}
